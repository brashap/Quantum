\documentclass{beamer}
\setbeamertemplate{navigation symbols}{}
\usepackage{comment}

\setbeamercolor{frametitle}{fg=black,bg=white}
\setbeamercolor{title}{fg=black,bg=yellow!85!orange}
\usetheme{AnnArbor}

\usepackage{textpos} % package for the positioning
\usepackage{listings}
\usepackage{xcolor}
\usepackage[most]{tcolorbox}
\usepackage{mathtools}
\usepackage{graphicx}
\usepackage{graphbox}
\usepackage{movie15}
\usepackage{caption}
\DeclareCaptionType{code}[Code Listing][List of Code Listings] 

\definecolor{codegreen}{rgb}{0,0.6,0}
\definecolor{codegray}{rgb}{0.5,0.5,0.5}
\definecolor{codepurple}{rgb}{0.58,0,0.82}
\definecolor{backcolour}{rgb}{0.95,0.95,0.92} 
\lstdefinestyle{mystyle}{
    backgroundcolor=\color{backcolour},   
    commentstyle=\color{codegreen},
    keywordstyle=\color{magenta},
    numberstyle=\tiny\color{codegray},
    stringstyle=\color{codepurple},
    basicstyle=\ttfamily\footnotesize,
    breakatwhitespace=false,         
    breaklines=true,                 
    captionpos=b,                    
    keepspaces=true,                 
    numbers=left,                    
    numbersep=5pt,                  
    showspaces=false,                
    showstringspaces=false,
    showtabs=false,                  
    tabsize=2
}

\lstset{style=mystyle}

\lstdefinelanguage
   [x64]{Assembler}     % add a "x64" dialect of Assembler
   [x86masm]{Assembler} % based on the "x86masm" dialect
   % with these extra keywords:
   {morekeywords={CDQE,CQO,CMPSQ,CMPXCHG16B,JRCXZ,LODSQ,MOVSXD, %
                  POPFQ,PUSHFQ,SCASQ,STOSQ,IRETQ,RDTSCP,SWAPGS, %
                  rax,rdx,rcx,rbx,rsi,rdi,rsp,rbp, %
                  r8,r8d,r8w,r8b,r9,r9d,r9w,r9b, %
                  r10,r10d,r10w,r10b,r11,r11d,r11w,r11b, %
                  r12,r12d,r12w,r12b,r13,r13d,r13w,r13b, %
                  r14,r14d,r14w,r14b,r15,r15d,r15w,r15b}} %


\beamersetuncovermixins{\opaqueness<1>{25}}{\opaqueness<2->{15}}

%Copyright
\addtobeamertemplate{frametitle}{}{%
\begin{textblock*}{50mm}(0cm,-1.25cm)
\color{yellow!85!orange}
\tiny{Copyright \copyright 2024 CNM.}
\end{textblock*}}

% position the logo
\addtobeamertemplate{frametitle}{}{%
\begin{textblock*}{100mm}(11.4cm,-1.3cm)
\includegraphics[height=1cm,width=1cm,keepaspectratio]{fig/ddclogotransparent.png}
\end{textblock*}}

\AtBeginSection[]{
  \begin{frame}
  \vfill
  \centering
  \begin{beamercolorbox}[sep=8pt,center,shadow=true,rounded=true]{title}
    \usebeamerfont{title}\insertsectionhead\par%
  \end{beamercolorbox}
  \vfill
  \end{frame}
}

\begin{document}
\title{Quantum Technician Bootcamp}
\author{Brian Rashap}
\date{September 2025} 

\begin{frame}
\titlepage
\end{frame}

\section{Introduction}

\begin{frame}\frametitle{Instructional team}
A physicist, an engineer, and a technician walk into a classroom...
\end{frame}

\begin{frame}
\frametitle{The Engineer: Brian Rashap, Ph.D.}
\begin{columns}
\begin{column}{6.5cm}
\begin{itemize}
\item Proud husband of Krista and father of Shelby (27) and Ethan (23)
\item Electrical Engineer (Michigan) with 23 years industrial experience (Intel)
\item Created and taught IoT Bootcamp for 5 years (and going)
\item Hobbies: painting, cycling, swimming, reading, spending time with family
\end{itemize}
\end{column}
\begin{column}{4.5cm}
\begin{center}
\includegraphics[width=4.5cm]{fig/rashapfamily.jpg}
\end{center}
\end{column}
\end{columns}
\end{frame}

\begin{frame}
\frametitle{The Physicist: Megan Ivory}
\begin{columns}
\begin{column}{8cm}
\begin{itemize}
\item Foster mom to two cats, Joey and Tessa; auntie to two humans, Liana and Azeria
\item PhD in atomic physics (William and Mary), startup ColdQuanta (now Infleqtion), Univ of Washington, Sandia since 2019 working on atomic clocks, quantum computers, and quantum programs like QuLL!
\item Outside hobbies: trail running, hiking, gardening, backpacking, trad climbing, canyoneering, and skiing
\item Inside hobbies: cooperative board games, reading trashy fantasy series, acting/writing
\end{itemize}
\end{column}
\begin{column}{3.5cm}
\begin{center}
\includegraphics[width=3.5cm]{fig/meganfam.jpg}
\end{center}
\end{column}
\end{columns}

\vspace{0.5cm}

I’ve never taken a quantum information science class…
\end{frame}

\begin{frame}
\frametitle{The Technician: Shawn Morales}
\begin{columns}
\begin{column}{6cm}
\begin{itemize}
\item Blessed husband of Joy and proud father of Katelyn (29) and Kimber (28)
\item Intel Manufacturing Technician of 32 years
\item Worked on 26 different tools within Intel
\item Adore my time with family, mountain biking, and the great outdoors
\end{itemize}
\end{column}
\begin{column}{5cm}
\begin{center}
\includegraphics[width=4.5cm]{fig/shawn.jpg}
\end{center}
\end{column}
\end{columns}
\end{frame}

\begin{frame}\frametitle{Student Introductions}
\begin{itemize}
\item Your name
\item What were you up to before starting the bootcamp
\item Any prior experience with math, science, machining, etc.
\item What do you hope to get out of the bootcamp
\end{itemize}
\end{frame}

\begin{frame}\frametitle{Drinks, Lunch and Snacks}
\begin{itemize}
\item There is a kitchenette down the hall with coffee/tea, snacks, frig, microwave
\begin{itemize}
\item The rest of the room is the TSL Lab - please don't disturb it
\item Please clean up the area every time you use it.
\item Don't leave leftovers overnight in the frig
\end{itemize}
\item Drink are allowed in the QuLL, but not near the optical tables or vacuum systems
\item You can eat lunch in the QuLL, 
\begin{itemize}
\item There will be a mid-day 1 hour lunch break
\item Again, no food/drink near optical tables or vacuum systems
\item Tables wiped down at the end of lunch
\end{itemize}
\item No food trash in QuLL garbage can (use the kitchenette trash)
\end{itemize}
\end{frame}


\begin{frame}\frametitle{Class Rules}
\begin{itemize}
\item Respect each other. Help each other.
\item Ask questions. 
\item Be on time (let us know via Slack if you won't be here) 
\item Keep your workspace and the classroom neat and tidy.
\item If you are struggling, let us know. We are here to HELP!
\item Class hours
\begin{itemize}
\item Mon-Th: 8am to 5pm \footnote{Doors open at 7:50, please be in your seats ready to learn by 8:00} and Friday: 8am to 3pm \footnote{Occasionally on Friday there will be optional activities from 3 to 5}
\item Lunch Break: 1 hour near noon. Maybe combined with work time. 
\item Please respect the instructors' lunch break as well.
\end{itemize}
\item Phone Policy: phones should not be out or used during class
\begin{itemize}
\item No gaming, no surfing social media
\item If you need to take/make a call, please set out of the classroom
\item Exceptions: two-factor authentication, pictures of projects, class videos
\end{itemize}
\end{itemize}
\end{frame}

\begin{frame}\frametitle{Credit for Prior Learning (CPL)}
If students enroll in an academic program at CNM, they are eligible for up to 24 credits though CPL:

\begin{itemize}
\item Current CPL (will be finalized before December)
\begin{itemize}
\item MATH 1220 College Algebra
\item BCIS 1110 Fundamentals of Information Literacy and Systems
\item BUSA Business Professionalism
\end{itemize}
\item CPL that can be claimed starting in Fall 2026
\begin{itemize}
\item ENGT 10xx Optics
\item ENGT 20xx Laser and Photonics
\item CSCI 10xx Survey of Quantum Computing
\item CSCI 20xx Quantum Hardware
\end{itemize}
\end{itemize}
\end{frame}

\begin{frame}\frametitle{Professional Development and Guest Speakers}
\begin{columns}
\begin{column}{5.5cm}
\begin{itemize}
\item Clifton Strength Finder
\item CNM Workforce and Community Success
\item LinkedIn / Online Job Search
\item Resume Workshop
\item Interview Skills
\item Effective Influencing
\item Constructive Confrontation
\item Presentation/Pitch Pitch
\end{itemize}
\end{column}
\begin{column}{5.5cm}
\begin{itemize}
\item City of ABQ
\item Elevate Quantum
\item Quantum NM Institute / UNM
\item Quantinuum
\item Qunnect
\item Vescent
\item 3D Glass Solutions
\item Infleqtion
\end{itemize}
\end{column}
\end{columns}
\end{frame}



\begin{frame}\frametitle{Foundation of Lean Certificate}
\begin{center}
\includegraphics[width=8cm]{fig/nmmep.png} 
\end{center}
\end{frame}


\begin{frame}\frametitle{IPC Solder Certification}
\begin{columns}
\begin{column}{5.5cm}
	\begin{center}	
	\includegraphics[width=5cm]{fig/ipc1.png} 
	\end{center}
\end{column}
\begin{column}{5.5cm}
	\begin{center}	
	\includegraphics[width=5cm]{fig/ipc2.jpg} 
	\end{center}
\end{column}
\end{columns}
\end{frame}



\begin{frame}\frametitle{Important Attribute in an Employee}
\begin{columns}
\begin{column}{5cm}
\begin{itemize}
\item Tolerance of Ambiguity
\item Attention to Detail
\item Reliability 
\item Curiosity
\item Structured Problem Solving
\end{itemize}
\end{column}
\begin{column}{5cm}
\begin{center}
\includegraphics[width=3.5cm]{fig/scimeth.jpg}
\end{center}

\end{column}
\end{columns}


\vspace{1cm}

Note: employers regularly ask the instructional team for references, it is these attributes that we will be the basis of our recommendation. 
\end{frame}

\section{Overview: QIS in NM}

\begin{frame}\frametitle{Introduction}
\begin{center}
\includegraphics[width=12cm]{fig/Slide2.jpeg}
\end{center}
\end{frame}

\begin{frame}\frametitle{Introduction}
\begin{center}
\includegraphics[width=12cm]{fig/Slide4.jpeg}
\end{center}
\end{frame}

\begin{frame}\frametitle{Introduction}
\begin{center}
\includegraphics[width=12cm]{fig/supercat.jpg}
\end{center}
\end{frame}

\begin{frame}\frametitle{Introduction}
\begin{center}
\includegraphics[width=12cm]{fig/Slide5.jpeg}
\end{center}
\end{frame}

\begin{frame}\frametitle{Introduction}
\begin{center}
\includegraphics[width=12cm]{fig/Slide6.jpeg}
\end{center}
\end{frame}

\begin{frame}\frametitle{Introduction}
\begin{center}
\includegraphics[width=12cm]{fig/Slide7.jpeg}
\end{center}
\end{frame}

\begin{frame}\frametitle{Introduction}
\begin{center}
\includegraphics[width=12cm]{fig/Slide8.jpeg}
\end{center}
\end{frame}

\begin{frame}\frametitle{Introduction}
\begin{center}
\includegraphics[width=12cm]{fig/Slide9.jpeg}
\end{center}
\end{frame}

\begin{frame}\frametitle{Introduction}
\begin{center}
\includegraphics[width=12cm]{fig/Slide10.jpeg}
\end{center}
\end{frame}

\begin{frame}\frametitle{Introduction}
\begin{center}
\includegraphics[width=12cm]{fig/Slide11.jpeg}
\end{center}
\end{frame}

\begin{frame}\frametitle{Introduction}
\begin{center}
\includegraphics[width=12cm]{fig/Slide12.jpeg}
\end{center}
\end{frame}

\begin{frame}\frametitle{Introduction}
\begin{center}
\includegraphics[width=12cm]{fig/Slide13.jpeg}
\end{center}
\end{frame}

\begin{frame}\frametitle{Introduction}
\begin{center}
\includegraphics[width=12cm]{fig/Slide14.jpeg}
\end{center}
\end{frame}

\begin{frame}\frametitle{Introduction}
\begin{center}
\includegraphics[width=12cm]{fig/Slide15.jpeg}
\end{center}
\end{frame}



\begin{frame}\frametitle{What you will learn}
\begin{itemize}
\item Optics
\item Lasers / Photonics
\item Ultra-High Vacuum Systems
\item Quantum Phenomenon
\item Applied Mathematics
\item Structured Problem Solving
\end{itemize}
\end{frame}


\begin{frame}\frametitle{Brightspace}
\begin{center}
\includegraphics[width=10cm]{fig/brightspace1.jpg}

\vspace{1cm}

\includegraphics[width=10cm]{fig/brightspace2.jpg}
\end{center}
\end{frame}


\begin{frame}\frametitle{Solidworks - Windows Users}


To install Solidworks (Windows only), go to \url{http://www.SolidWorks.com/SEK}
\begin{itemize}
\item Enter your contact information.
\item Check the radio button “Yes” under "I already have a Serial Number that starts with 9020".
\item Select the version: 2024 SP5.0 and click Request Download.
\item On the next page, Accept the agreement and continue.
\item On the final page, click the Download button to download the SolidWorks Installation Manager.
\item Unzip the files to launch the Installation.
\item Select the option for Individual/On this machine.
\item Install using the following serial number provided by your instructor.
\end{itemize}



\end{frame}

\begin{comment}

macOS and Linux users will use onShape: \url{https://www.onshape.com/en/education/}

2021-2022: 9020005002143476M2TGM8G4
2022-2023: 9020005328617363VNZVR37D


1.  Uninstall any/all previous version of SOLIDWORKS
2.  Go to www.solidworks.com/SEK
3.  Use SEK-ID = XSEK12 and enter all relevant information into the form
4.  Choose 2023 for the most recent version of SOLIDWORKS
5.  You will then be asked to enter the school’s unique SEK Serial Number:
Your Student Download Serial Number is:

9020005723476091HT7RDP86 - (Qty: 60)
9020005723816337C39Z2D27 - (Qty: 20)

* Follow all remaining instructions and accept all defaults
* Your VAR Name is GoEngineer

\end{comment}

\begin{frame}\frametitle{Other Software}
\begin{enumerate}
\item Bambu Studio
	\begin{itemize}
	\item \url{https://bambulab.com/en/download/studio}
	\end{itemize}
\item Microsoft Excel
	\begin{itemize}
	\item \url{https://office365.com}
	\item Log in with your CNM credentials
	\item Download excel (and any other office programs) 
	\end{itemize}
\item Adobe Illustrator 
	\begin{itemize}
	\item \url{https://adobe.com/creativecloud}
	\item Log in with your CNM credentials
	\item Select Work/School account 
	\end{itemize}
\item Octave
	\begin{itemize}
	\item \url{https://octave.org/download}
	\end{itemize}
\item Bookmark the following: 
	\begin{itemize}
	\item \url{https://www.desmos.com}
	\item \url{https://quantum.cloud.ibm.com/composer}
	\item \url{https://colab.research.google.com}
	\end{itemize}
\end{enumerate}
\end{frame}

\begin{frame}\frametitle{Inaugural Cohort}
\begin{center}
\includegraphics[width=10cm]{fig/underconstruction.jpg}
\end{center}
\end{frame}

\begin{frame}\frametitle{Pre Knowledge Check}
\begin{center}
\includegraphics[width=8cm]{fig/precat.jpg}
\end{center}

\vspace{0.25cm}
\begin{itemize}
\item We are assuming no prior knowledge
\item Pre and Post check will be used to help instructional team for future cohorts
\item D) I am not familiar with this concept is an option on every question
\end{itemize}


\end{frame}



\section{Safety}

\begin{frame}\frametitle{Safety Walk}
\begin{center}
\includegraphics[width=6cm]{fig/gemba.png}
\end{center}
\end{frame}

\begin{frame}\frametitle{Laser Safety - Laser Classes}
\begin{center}
\includegraphics[width=10cm]{fig/lasersafe.jpg}
\end{center}
\end{frame}

\begin{frame}\frametitle{Laser Safety - Class 2}
\begin{center}
\includegraphics[width=10cm]{fig/lsafe.png}
\end{center}
\begin{itemize}
\item Class 2 lasers, which are limited to 1 mW of visible continuous-wave radiation, are safe because the blink reflex will limit the exposure in the eye to 0.25 seconds. This category only applies to visible radiation (400 - 700 nm).
\end{itemize}
\end{frame}

\begin{frame}\frametitle{Laser Safety - Class 3R}
\begin{center}
\includegraphics[width=10cm]{fig/lsafe.png}
\end{center}
\begin{itemize}
\item Class 3R lasers produce visible and invisible light that is hazardous under direct and specular-reflection viewing conditions. Eye injuries may occur if you directly view the beam, especially when using optical instruments. Lasers in this class are considered safe as long as they are handled with restricted beam viewing. Visible, continuous-wave lasers in this class are limited to 5 mW of output power.
\end{itemize}
\end{frame}

\begin{frame}\frametitle{Laser Safety - Class 3R}
\begin{center}
\includegraphics[width=10cm]{fig/lsafe.png}
\end{center}
\begin{itemize}
\item Class 3B lasers are hazardous to the eye if exposed directly. Diffuse reflections are usually not harmful. Safe handling of devices in this class includes wearing protective eyewear where direct viewing of the laser beam may occur. Lasers of this class must be equipped with a key switch, laser safety signs should be used. Laser products with power output near the upper range of Class 3B may also cause skin burns.
\end{itemize}
\end{frame}

\begin{frame}\frametitle{Lock Out Tag Out}
\begin{center}
\includegraphics[width=4cm]{fig/loto.jpg}
\end{center}

\begin{itemize}
\item Electrical Energy
\item Mechanical Energy
\item Mechanical Energy - Pneumatics
\end{itemize}
\end{frame}

\begin{frame}\frametitle{Sharps}
\begin{center}
\includegraphics[width=6cm]{fig/sharps.png}
\end{center}
\end{frame}


\begin{frame}\frametitle{High Voltage}
\begin{center}
\includegraphics[width=6cm]{fig/highV.jpg}
\end{center}
\end{frame}

\section{Python}

\begin{frame}\frametitle{Google Colab and Jupyter \footnote{formerly ipython or interactive python} Notebooks}
\begin{columns}
\begin{column}{4cm}
\begin{center}
\includegraphics[width=3.5cm]{fig/colab.png}
\includegraphics[width=4cm]{fig/jupyter.png}
\end{center}
\end{column}
\begin{column}{7cm}
\begin{itemize}
\item Colab is a hosted Jupyter Notebook service that requires no setup to use and provides free access to computing resources, including GPUs and TPUs. Colab is especially well suited to machine learning, data science, and education.
\item Jupyter Notebook is an interactive web application for creating and sharing computational documents. It supports more than 40 languages including Python, R, and Scala.
\end{itemize}
\end{column}
\end{columns}
\end{frame}

\begin{frame}\frametitle{Python}
\begin{columns}
\begin{column}{3.5cm}
\begin{center}
\includegraphics[width=3.5cm]{fig/python1.png}
\end{center}
\end{column}
\begin{column}{7.5cm}
\begin{itemize}
\item Easy to understand: Python’s readability, non-complexity and ability for fast prototyping contribute to its popularity. 
\item Libraries: Python comes with numerous built-in libraries that are well suited for data mining, machine learning and AI.
\item Easier, more powerful implementation: With Python, programmers spend less time writing code and debugging errors compared with other languages.
\item Syntax: Python uses modern scripting and friendly syntax.
\end{itemize}
\end{column}
\end{columns}


\end{frame}

\begin{frame}\frametitle{Python Libraries}
Python libraries are collections of pre-written code, including modules and routines, that provide specific functionality to developers. 
\begin{itemize}
\item Numpy: fundamental package for scientific computing in Python. It is a Python library that provides a multidimensional array objects. It contains routines for operations on arrays, including mathematical, logical, shape manipulation,  basic linear algebra, basic statistical operations, random simulation, etc.
\item Scipy: algorithms for optimization, integration, interpolation, eigenvalue problems, algebraic equations, differential equations, statistics and many other classes of problems.
\item Matplotlib: Matplotlib is a comprehensive library for creating static, animated, and interactive visualizations in Python.
\end{itemize}

Getting Started: 
\url{https://colab.research.google.com/github/data-psl/lectures2020/blob/master/notebooks/01_python_basics.ipynb}
\end{frame}

\section{Atomic Quantum Systems}

\section{Vacuum Systems}


\section{Geometric Optics}

\begin{frame}\frametitle{Ray Nature of Light}
The word "ray" means a straight line that originates at some point.

\begin{center}
\includegraphics[width=6cm]{fig/rays.jpg}
\end{center}

The part of optics dealing with the ray aspect of light is called "geometric optics."

\end{frame}


\begin{frame}\frametitle{Reflection}
The angle of reflection equals the angle of incidences

\begin{center}
\includegraphics[width=6cm]{fig/reflect.png}
\end{center}

\end{frame}

\begin{frame}\frametitle{Demonstration: Handling Optics}
\begin{center}
\includegraphics[width=6cm]{fig/opticscleaning.jpg}
\end{center}
\end{frame}

\begin{frame}\frametitle{Assignment: Laser Tag}
\begin{columns}
\begin{column}{5cm}
\begin{center}
\includegraphics[width=4.5cm]{fig/lasertag.jpg}
\end{center}
\end{column}
\begin{column}{7cm}
\begin{itemize}
\item Align a Laser and a target at opposite ends of the table.
\item When obstacles are placed in the path, without adjusting laser, add mirrors to get the beam to the target.
\end{itemize}
\end{column}

\end{columns}
\end{frame}




\begin{frame}\frametitle{Rough vs Smooth Surfaces}

\begin{center}
\includegraphics[width=3.5cm]{fig/reflect_rough.png}
\includegraphics[width=3.5cm]{fig/reflect_paper.png}
\includegraphics[width=3.5cm]{fig/reflect_mirror.png}
\end{center}

\end{frame}

\begin{frame}\frametitle{Mirrors and Virtual Images}
When we see ourselves in a mirror, it appears that our image is actually behind the mirror.

\begin{center}
\includegraphics[width=6cm]{fig/virtual.png}
\end{center}

\end{frame}

\begin{frame}\frametitle{Speed of Light}
\begin{columns}
\begin{column}{7cm}
\begin{itemize}
\item In 1676, Danish astronomer Ole Roemer noted the change in orbital period of Jupiter's moons depending of if the earth was moving towards or away from Jupiter. He as able to calculate speed of light to be $2.26 x 10^8 (\frac{m}{s})$.
\item In 1887, American physicist Albert Michelson used a rotating mirror to get a more precise measurement of the speed of light. 
\item Today, the speed of light is known as:
\end{itemize}
\begin{center}
$c=2.99792458×10^8 (\frac{m}{s})$.
\end{center}

\end{column}
\begin{column}{5cm}
\begin{center}
\includegraphics[width=4.5cm]{fig/c_meas.png}
\end{center}
\end{column}
\end{columns}
\end{frame}

\begin{frame}\frametitle{Assignment: Two Mirror Walk}
\begin{columns}
\begin{column}{4cm}
\begin{center}
\includegraphics[width=3cm]{fig/twoMirror.jpg}

\vspace{1cm}
\includegraphics[width=4cm]{fig/twoMirror2.jpg}

\end{center}
\end{column}
\begin{column}{7.5cm}
\begin{enumerate}
\item Setup: Laser, $45^{\circ}$ mirror, second $45^{\circ}$ mirror, two irises, target.
\item Adjust first mirror to center beam on center iris 1
\item Open iris 1, adjust second mirror to center beam on iris 2
\item Iterate steps 2 and 3 until the beam passes through the center of both iris and hits target.
\end{enumerate}
\end{column}
\end{columns}

\end{frame}

\section{Math Interlude: Trigonometry}

\begin{frame}\frametitle{Algebraic Functions}
\begin{columns}
\begin{column}{4.5cm}
An algebraic function provides a "y-value" for every "x-value"
\begin{itemize}
\item Linear: $y = x + 2$
\item Quadratic: $y = x^2$
\item Periodic: $y = sin(x)$
\end{itemize}
\end{column}
\begin{column}{7cm}
\begin{center}
\includegraphics[width=7cm]{fig/basicfun.jpg}
\end{center}
\end{column}
\end{columns}
\end{frame}

\begin{frame}\frametitle{More Desmos Fun}

\begin{center}
\includegraphics[width=12cm]{fig/desmosfun.jpg}
\end{center}
\end{frame}


\begin{frame}\frametitle{Pi ($\pi)$}

\begin{center}
\includegraphics[width=8cm]{fig/pi.png}
\end{center}

\end{frame}


\begin{frame}\frametitle{Unit Circle and Trigonometric Functions}
The Unit Circle is a circle with a radius of 1.
\begin{columns}
\begin{column}{6cm}
\begin{center}
\includegraphics[scale=0.75]{fig/unitcircle.png}
\end{center}
\end{column}
\begin{column}{6cm}
\begin{center}
\includegraphics[scale=0.75]{fig/unitcircle_cst.png}
\end{center}
\end{column}
\end{columns}
The Unit Circle can be used to map out the trigonometric values of sine, cosine, and tangent.
\end{frame}


\begin{frame}\frametitle{Unit Circle and the Value of $sin(\theta)$}
\begin{columns}
\begin{column}{6cm}
\begin{center}
\includegraphics[scale=0.25]{fig/unitcircle_sin.png}
\end{center}
\end{column}
\begin{column}{6cm}
\begin{center}
\includegraphics[scale=0.75]{fig/unitcircle_rad.png}
\end{center}
\end{column}
\end{columns}

\vspace{0.25cm}

\begin{itemize}
\item $sin(\theta)$ is the y-value of the point on the Unit Circle at angle $\theta$. 
\item In our trig functions, $\theta$ is measured in radians (rad), not degrees.
\item 360 degrees = $2 \pi$ radians.
\end{itemize}
\end{frame}



\begin{frame}\frametitle{Sine Waves}
\begin{center}
\includegraphics[scale=0.25]{fig/sin.jpg}
\end{center}

\begin{center}
$y = A * sin(2*\pi*\nu*t) + B$
\end{center}

where A = amplitude, B = offset, $\nu$ = frequency = $\frac{1}{period}$, 

and t = time in seconds.
\end{frame}


\begin{frame}\frametitle{Using Desmos (desmos.com/calculator)}
\begin{center}
\includegraphics[scale=0.35]{fig/desmos1.jpg}
\end{center}
\begin{center}

\vspace{0.5cm}

\includegraphics[scale=0.35]{fig/desmos2.jpg}
\end{center}
\end{frame}

\begin{frame}\frametitle{Phase Shift}
The sine wave can be shifted relative to each other by adding in a phase shift ($\phi$), which will shift the wave to the left or right.

Blue lags Red: 
\begin{center}
\includegraphics[height=2.6cm]{fig/sin_lag.jpg}
\end{center}

Green leads Red: 
\begin{center}
\includegraphics[height=2.6cm]{fig/sin_lead.jpg}
\end{center}

\end{frame}




\begin{frame}\frametitle{SOH CAH TOA}

\begin{itemize}
\item sin = opposite over hypotenuse
\item cos = adjacent over hypotenuse
\item tan = opposite over adjacent
\end{itemize}

\vspace{0.25cm}

\begin{center}
\includegraphics[scale=0.35]{fig/sohcahtoa.jpg}
\end{center}
\end{frame}

\begin{frame}\frametitle{Assignment: Calculating Angles: Laser Billards}
\begin{columns}
\begin{column}{4cm}
\begin{center}
\includegraphics[width=4cm]{fig/billards.jpg}
\end{center}
\end{column}
\begin{column}{7cm}
\begin{itemize}
\item Setup a mirror line on the optical table.
\item Place the laser and target two different distances from the line.
\item Calculate the location of the mirror, place the mirror there.
\item Calculate the angle of the laser, adjust laser angle.
\item Turn on laser and see how close on target the calculations are.
\end{itemize}
\end{column}
\end{columns}
\end{frame}


\section{Return to Geometric Optics}

\begin{frame}\frametitle{Refraction}
The changing of a light ray’s direction (loosely called bending) when it passes through variations in matter is called refraction.

\begin{center}
\includegraphics[width=5cm]{fig/fish.png}
\end{center}

\end{frame}


\begin{frame}\frametitle{Index of Refraction}

\begin{columns}
\begin{column}{6.5cm}
The speed of flight depends strongly on the type of material. We define the index of refraction ($n$) as

\begin{center}
$n = \frac{c}{v}$
\end{center}

where $v$ is the speed of light in the material and $c$ is the speed of light in a vacuum.
\end{column}
\begin{column}{5cm}
\begin{center}
\includegraphics[width=5cm]{fig/n_table.png}
\end{center}
\end{column}
\end{columns}
\end{frame}

\begin{frame}\frametitle{Law of Refraction - Snell's Law}
The law of refraction is also called Snell’s law after the Dutch mathematician Willebrord Snell (1591–1626).

\begin{center}
\includegraphics[width=6cm]{fig/snell.png}
\end{center}

\begin{center}
Snell's Law: $n_1 \sin{\theta_1} = n_2 \sin{\theta_2}$
\end{center}


\end{frame}

\begin{frame}\frametitle{Finding Index of Refraction}

Snells Law:

\begin{equation}
n_1 \sin{\theta_1} = n_2 \sin{\theta_2}
\end{equation}

Rearranging to isolate $n_2$:

\begin{equation}
n_2 = n_1 \frac{\sin{\theta_1}}{\sin{\theta_2}}
\end{equation}

For example, if the initial medium is air, $\theta_1 = 30^\circ \textdegree$ and $\theta_2 = 22^\circ$

\begin{equation}
n_2 = (1.00) \cdot \frac{\sin{30^\circ}}{\sin{22^\circ}} = \frac{0.500}{0.375} = 1.33
\end{equation}

\end{frame}

\begin{frame}\frametitle{Assignment: Measuring Refraction}
\begin{columns}
\begin{column}{4cm}
\begin{center}
\includegraphics[width=4cm]{fig/refract9.jpg}
\end{center}
\end{column}
\begin{column}{7cm}
\begin{itemize}
\item Some assignment on refraction
\item Acrylic, water, what else?
\item Different color lasers (red, green, and if we get blue in time)
\end{itemize}
\end{column}
\end{columns}
\end{frame}


\begin{frame}\frametitle{Total Internal Reflection}

\begin{columns}
\begin{column}{9.5cm}
Good mirrors reflect $>90$\% of the light; however, total reflection can be produced via refraction.\newline

If the index of refraction of the second medium is less than that of the first medium, the rays are refracted away from the perpendicular. 

\begin{itemize}
\item Since $n_1 > n_2$, the angle of refraction is greater than the angel of incidence: $\theta_2 > \theta_1$.
\item Increasing $\theta_1$ causes $\theta_2$ to increase.
\item The critical angle ($\theta_c$) is defined to be the incident angle ($\theta_1$) that produces a $\theta_2 = 90^\circ$ 
\end{itemize}

The critical angle is given by:
\begin{equation}
\theta_c = sin^{-1}(\frac{n_2}{n_1}), \hspace{3mm} \text{for} \hspace{2mm}  n_1 > n_2
\end{equation}

\end{column}
\begin{column}{3cm}
\begin{center}

\includegraphics[width=2.5cm]{fig/tir1.png}

\includegraphics[width=2.5cm]{fig/tir2.png}

\includegraphics[width=2.5cm]{fig/tir3.png}

\end{center}
\end{column}
\end{columns}
\end{frame}


\begin{frame}\frametitle{Fiber Optic Cable}
The fiber optic cable takes advantage of the core having a high index of refraction than the cladding.

\begin{center}
\includegraphics[width=6cm]{fig/cladding.jpg}
\end{center}

\vspace{3cm}

More to come on fibers as we progress through the Quantum Journey
\end{frame}

\begin{frame}\frametitle{Fiber: Acceptance Angle}
For multi-mode fibers, the numerical appeture (NA) provides  a good estimate of the maximum acceptance angle. 
\begin{columns}
\begin{column}{7cm}


\begin{itemize}
\item The cutoff angle is the maximum acceptance angle ($\theta_{max}$), which is related to NA:
\[ NA = n_0 sin(\theta_{max}) = \sqrt{n_{core}^2 + n_{clad}^2} \]

\item Rays with an angle of incidence $\leq \theta_{max}$ are totally internally reflected (TIR) at the fiber core/cladding boundary.

\item Rays with an angle of incidence $> \theta_{max}$ refract at and pass through the boundary.

\end{itemize}

\end{column}
\begin{column}{4cm}
\begin{center}

\includegraphics[width=3.8cm]{fig/fiber_NA.png}

\vspace{0.25cm}

\includegraphics[width=3.8cm]{fig/fiber_TIR.png}

\end{center}
\end{column}
\end{columns}
\end{frame}


\begin{frame}\frametitle{Assignment: Fiber Coupling}
\begin{columns}
\begin{column}{4.5cm}
\begin{center}
\includegraphics[width=4.5cm]{fig/coupling.jpg}

\vspace{1cm}

\includegraphics[width=3cm]{fig/pm100d.jpg}

\end{center}
\end{column}
\begin{column}{6.5cm}
\begin{itemize}
\item Use the two mirror walk to align a HeNe laser beam to a fiber.
\item Improve the quality of the coupling by maximizin the power meter reading.
\item How can you improve the coupling? 
\end{itemize}


\end{column}
\end{columns}

\end{frame}



\begin{frame}\frametitle{Dispersion}
Dispersion is defined to be the spreading of white light into its full spectrum of wavelengths.

\begin{center}
\includegraphics[width=10cm]{fig/rainbow.jpg}
\end{center}

\begin{itemize}
\item The angle of refraction depends on the index of refraction.
\item The index of refraction (n) depends on the properties of the medium.
\item However, for a given medium, n also depends on the optical wavelength.
\end{itemize}


\end{frame}

\begin{frame}\frametitle{Index of Refraction by Wavelength}
Index of refraction ($n$) by wavelength ($\lambda$):

\begin{center}
\includegraphics[width=10cm]{fig/nbylambda.jpg}
\end{center}
\end{frame}


\begin{frame}\frametitle{Glass Prism}
\begin{columns}
\begin{column}{5.5cm}
\begin{center}
\includegraphics[height=3.6cm]{fig/gpideal.jpg}
\end{center}
\end{column}
\begin{column}{5.5cm}
\begin{center}
\includegraphics[height=3.7cm]{fig/gpreal.jpg}
\end{center}
\end{column}
\end{columns}
\end{frame}

\begin{frame}\frametitle{Rainbow}
\begin{center}
\includegraphics[width=6cm]{fig/rainbow2.jpg}
\end{center}
Rainbows are produced by a combination of refraction and reflection. You may have noticed that you see a rainbow only when you look away from the sun. Light enters a drop of water and is reflected from the back of the drop. The light is refracted both as it enters and as it leaves the drop. Since the index of refraction of water varies with wavelength, the light is dispersed, and a rainbow is observed.
\end{frame}

\begin{frame}\frametitle{Rainbow as an Arc}
\begin{columns}
\begin{column}{5.5cm}
\begin{center}
\includegraphics[width=5cm]{fig/rainbow3.jpg}
\end{center}
\end{column}
\begin{column}{5.5cm}
\begin{center}
\includegraphics[width=5cm]{fig/rainbow4.jpg}
\end{center}
\end{column}
\end{columns}
\end{frame}


\section{Lens}

\begin{frame}\frametitle{Lens}
With the Law of Refraction, we can explore the properties of lens and how images are formed.
\begin{columns}
\begin{column}{6.5cm}
\begin{itemize}
\item The word lens comes from the Latin word for lentil bean, the shape of which is similar to a convex lens.
\item Convex Lens: all light rays that enter parallel to the axis cross one another at  a single point on the opposite side of the lens, i.e., they converge.
\item Concave Lens: all light rays that enter parallel to the axis diverge (bend away) from the lens axis.
\end{itemize}
\end{column}
\begin{column}{4.5cm}
\begin{center}
\includegraphics[width=4cm]{fig/cavevex.jpg}
\end{center}
\end{column}
\end{columns}
\end{frame}


\begin{frame}\frametitle{Convex Lens}
With the Law of Refraction, we can explore the properties of lens and how images are formed.
\begin{columns}
\begin{column}{6cm}
\begin{itemize}
\item A ray of light bends (refracts) at both interface, and for convex lens converge.
\item The point at which the rays crossed is defined as the Focal point ($F$) of the lens.
\item The distance from the center of the lens to its focal point is called the focal length ($f$).
\item The Power of the lens, measuring in Diopters ($P=\frac{1}{f}$) where $f$ is measured in meters.
\end{itemize}
\end{column}
\begin{column}{5cm}
\begin{center}
\includegraphics[width=4cm]{fig/convex1.jpg}

\vspace{0.25cm}
\includegraphics[width=4cm]{fig/convex2.jpg}
\end{center}
\end{column}
\end{columns}
\end{frame}

\begin{frame}\frametitle{Concave Lens}
\begin{columns}
\begin{column}{6cm}
\begin{itemize}
\item A concave lens is a diverging lens, it causes light rays to bend away from the axis.
\item In the case of all rays entering parallel to its axis, the light appears to originate at the same point $F$.
\item The distance from the center of the lens to its focal point is called the focal length ($f$) and is defined to be negative.
\end{itemize}
\end{column}
\begin{column}{5cm}
\begin{center}
\includegraphics[width=4cm]{fig/concave1.jpg}

\vspace{0.25cm}
\includegraphics[width=4cm]{fig/concave2.jpg}
\end{center}
\end{column}
\end{columns}
\end{frame}


\begin{frame}\frametitle{Assigment: Find the Focal Length}
\begin{columns}
\begin{column}{4.6cm}
\begin{center}
\includegraphics[width=4.5cm]{fig/sunfire.jpg}
\end{center}
\end{column}
\begin{column}{7.4cm}
\begin{itemize}
\item Using some random lens, use the overhead lighting to find the focal length
\item Repeat on optical table using laser
\item Clean lens before putting away
\end{itemize}
\end{column}
\end{columns}
\end{frame}

\begin{frame}\frametitle{Thin Lens}
A thin lens is defined to be one whose thickness allows rays to refract but does not allow properties such as dispersion and aberrations.
\end{frame}

\begin{frame}\frametitle{Ray Tracing}
\begin{columns}
\begin{column}{7.4cm}
\begin{enumerate}
\item A ray entering a converging lens parallel to its axis passes through the focal point F of the lens on the other side.
\item A ray entering a diverging lens parallel to its axis seems to come from the focal point F.
\item A ray passing through the center of either a converging or a diverging lens does not change direction.
\item A ray entering a converging lens through its focal point exits parallel to its axis.
\item A ray that enters a diverging lens by heading toward the focal point on the opposite side exits parallel to the axis.
\end{enumerate}


\end{column}
\begin{column}{4.6cm}
\begin{center}
\includegraphics[width=4.5cm]{fig/raytrace.jpg}
\end{center}
\end{column}
\end{columns}
\end{frame}


\begin{frame}\frametitle{Image Formation}

\begin{center}
\includegraphics[width=8cm]{fig/imageform1.jpg}
\end{center}


Thin Lens equations:

\begin{columns}
\begin{column}{5.5cm}
\[ \frac{1}{d_o} + \frac{1}{d_i} = \frac{1}{f}\]
\end{column}
\begin{column}{5.5cm}
\[ \frac{h_i}{h_o} = -\frac{d_i}{d_o} = m\]
\end{column}
\end{columns}

\end{frame}


\begin{frame}\frametitle{Image Formation - Real Image}
\begin{columns}
\begin{column}{5.5cm}
\begin{center}
\includegraphics[width=5cm]{fig/imageform2.jpg}
\end{center}
\end{column}
\begin{column}{5.5cm}
\begin{center}
\includegraphics[width=5cm]{fig/imageform3.jpg}
\end{center}
\end{column}
\end{columns}

\vspace{1cm}

The image in which light rays from one point on the object actually cross at the location of the image and can be projected onto a screen, a piece of film, or the retina of an eye is called a real image.

\end{frame}

\begin{frame}\frametitle{Assignment - Focus CCD Camera}
\begin{columns}
\begin{column}{4.5cm}
\begin{center}
\includegraphics[width=4cm]{fig/thorCCD.jpg}
\end{center}
\end{column}
\begin{column}{7cm}
\begin{itemize}
\item Get a Thor CCD camera from the Optical Tweezer kit
\item Select lens tube, adjustable lens tube, and lens
\item Assemble and mount on table. 
\item Connect to ThorCam software
\item Using adjustable lens tube, focus on various objects
\end{itemize}
\end{column}
\end{columns}
\end{frame}

\begin{frame}\frametitle{Optical System Parameters}
\begin{columns}
\begin{column}{4.5cm}
\begin{center}
\includegraphics[width=4cm]{fig/na.jpg}
\includegraphics[width=4cm]{fig/dof.jpg}
\end{center}
\end{column}
\begin{column}{7cm}
\begin{itemize}
\item Numerical Aperture (NA):
\[NA = n \cdot sin(\alpha) \text{ where } \alpha = \frac{\theta}{2}\]
\item \textit{f}-number - light per unit area reaching image plane:
\[ \textit{f} / \# = \frac{f}{D} \approx \frac{1}{2 NA} \]
\item Resolution and Depth of Focus:
\[ R = \frac{1.22 \lambda}{2 NA}  \text{ and } DOF = \frac{ \lambda n}{NA^2} \]
\end{itemize}
\end{column}
\end{columns}
\end{frame}

\begin{frame}\frametitle{Aside: Camera \textit{f}-step}
\begin{center}
\includegraphics[width=10cm]{fig/aperture.jpg}
\end{center}
\[ \textit{f}/1 = \frac{f}{(\sqrt{2})^0}, \textit{f}/1.4 = \frac{f}{(\sqrt{2})^1}, \textit{f}/2 = \frac{f}{(\sqrt{2})^2}, \textit{f}/2.8 = \frac{f}{(\sqrt{2})^3},...\] 
\end{frame}



\begin{frame}\frametitle{Image Formation - Virtual Image}

\begin{center}
\includegraphics[width=8cm]{fig/imageform4.jpg}
\end{center}

If an object is held closer to the converging lens than its focal length ($f$), then the rays from a common point continue to diverge after passing through the lens. They all appear to originate from a point at the location of the image, on the same side of the lens as the object. This is a virtual image. 


\end{frame}

\begin{frame}\frametitle{Image Formation - Concave Lens}

\begin{center}
\includegraphics[width=8cm]{fig/imageform5.jpg}
\end{center}
\end{frame}


\begin{frame}\frametitle{Beam Expander/Reducer - Telescope}
\begin{columns}
\begin{column}{5.5cm}
\begin{center}
\includegraphics[width=5cm]{fig/beamXkepler.png}
Keplerian Design
\end{center}
\end{column}
\begin{column}{5.5cm}
\begin{center}
\includegraphics[width=5cm]{fig/beamXgalileo.png}
Galilean Design
\end{center}
\end{column}
\end{columns}

\vspace{0.25cm}
Both the bean's waist ($2W_0$) and the divergence angle ($\theta$) are affected by by the beam expanders and reducers. If Lens 2 is the output lens, then the beam expansion ratio ($m_{12}$) is: 
\begin{equation}
m_{12} = \frac{f_2}{f_1}
\end{equation}

\end{frame}



\section{Math Interlude: Probability}

\begin{frame}\frametitle{Probability and Statistics}
\begin{columns}
\begin{column}{5cm}
\begin{center}
\includegraphics[width=5cm]{fig/cartoonguidestats.jpg}
\end{center}
\end{column}
\begin{column}{5cm}
\begin{itemize}
\item Data Analysis: The gathering, display, and summary of data
\item Probability: The laws of chance (inside and outside of a casino)
\item Statistical Inference: the science of drawing statistical conclusions from specific data using the laws of probability
\end{itemize}
\end{column}
\end{columns}
\end{frame}

\begin{frame}\frametitle{A Tale as Old as Time}
\begin{center}
\includegraphics[width=6cm]{fig/vegas.jpg}
\end{center}


Gambling is as old as mankind, so it seems that probability should be almost as old. But, the realization that one could predict an outcome to a certain degree of accuracy was unconceivable until the $16^{th}$ century. In order to make a profit, underwriters were in need of dependable guidelines by which a profit could be expected, while the gambler was interested in predicting the possibility of gain.
\end{frame}



\begin{frame}\frametitle{Rich Guys Gambling}
\begin{columns}
\begin{column}{4cm}
\begin{center}
\includegraphics[width=3cm]{fig/cardano.jpg}
\end{center}
\end{column}
\begin{column}{7.5cm}
Known as the “Father of Probability”, Gerolamo Cardano was an Italian mathematician, physician, and gambler who first talked about probability. Cardano’s fascination with games of chance led him to write the first book dedicated to probability, “Liber de Ludo Aleae” (Book on Games of Chance), published in 1564. Cardano introduced concepts like odds and probabilities in this work, providing a framework for analyzing the likelihood of different outcomes in dice rolls and other games.
\end{column}
\end{columns}
\end{frame}

\begin{frame}\frametitle{Probability}


\begin{columns}
\begin{column}{5cm}
The probability of an event expresses the likelihood of the event outcome

\[P(event) = \frac{\text{\# of favorable outcomes}}{\text{\# of all possibly outcomes}}\]
\end{column}
\begin{column}{5cm}
\begin{center}
\includegraphics[width=3cm]{fig/coinflip.jpg}
\end{center}
\end{column}
\end{columns}
\end{frame}

\begin{frame}\frametitle{Frequency Tables and Histograms}

\begin{center}

Roll two dice and add them together, repeat. \newline

\includegraphics[width=10cm]{fig/stat1.jpg}
\end{center}

\end{frame}

\begin{frame}\frametitle{Assignment: Tenzi}
\begin{columns}
\begin{column}{4cm}
\begin{center}
\includegraphics[width=4cm]{fig/dice.jpg}
\end{center}
\end{column}
\begin{column}{7cm}

Role two dice 12 times, after each role:
\begin{itemize}
\item Record the sum
\item Record the running average (total up to this point divided by number of roles)
\end{itemize}
After the twelfth role:
\begin{itemize}
\item Create a histogram of the sums
\item Create a line chart of the running averages
\end{itemize}
\end{column}
\end{columns}
\end{frame}


\begin{frame}\frametitle{Summary Statistics}


\begin{columns}
\begin{column}{7cm}
\begin{itemize}
\item Mean (or average): add the totals and divide by number of samples
\[ \bar{x} = \frac{x_1 + x_2 + x_3 + ... + x_n}{n} \]
or
\[ \bar{x} = \frac{\sum_{i=1}^{n} (x_i)}{n}\]
\item Median: middle sample in the ordered distribution
\begin{itemize}
\item Odd: median is the value of middle sample
\item Even: median is the average of the two middle samples
\end{itemize}
\item Mode: the value the appears most often
\end{itemize}
\end{column}
\begin{column}{4cm}
\begin{center}
\includegraphics[width=4cm]{fig/stat2.jpg}


average = $\bar{x} = 7$

median $= 6$

mode $ = 6$
\end{center}
\end{column}
\end{columns}
\end{frame}


\begin{frame}\frametitle{Summary Statistics: Variation}
\begin{center}
\includegraphics[width=11cm]{fig/stat3.jpg}
\end{center}
\begin{itemize}
\item Interquartile Range (IRQ): Spread of the middle $50\%$ of the data
\begin{itemize}
\item Find the median
\item Find the first quartile (Q1): median of the lower half.
\item Find the third quartile (Q3): median of the upper half.
\item IRQ = Q3 - Q1 = 6
\end{itemize}


\item Standard Deviations ($\sigma$):
\[ \sigma = \sqrt{ \frac{ \sum_{i=0}^n (x_i - \bar{x})^2}{n}} = 3.06\]

\end{itemize}

\end{frame}


\begin{frame}\frametitle{Gaussian Distribution}

\begin{center}
\includegraphics[width=6cm]{fig/stat4.png}
\end{center}

Probability Distribution Function $f(x)$:

\[ f(x) = \frac{1}{\sqrt{2 \pi \sigma^2}} e^{\frac{-(x-\bar{x})^2}{2 \sigma^2}} \]

\end{frame}

\begin{frame}\frametitle{Gaussian Distribution: $3 \sigma$}
\begin{columns}
\begin{column}{6cm}
\begin{center}
\includegraphics[width=6cm]{fig/stat5.png}
\end{center}
\end{column}
\begin{column}{4cm}

\begin{center}
Galton Baord
\includegraphics[width=4cm]{fig/stat7.png}
\end{center}
\end{column}
\end{columns}
\end{frame}

\begin{frame}\frametitle{Assignment: Measuring Bean Profile}
\begin{columns}
\begin{column}{4cm}
\begin{center}
\includegraphics[width=3cm]{fig/profile1.jpg}
\includegraphics[width=3cm]{fig/profile2.jpg}
\end{center}
\end{column}
\begin{column}{7cm}
The laser beam has a Gaussian profile.
\begin{enumerate}
\item Direct a laser beam at a power meter, setting the correct wavelength
\item Moving the knife edge across the beam path, record the drop in power vs distance.
\item Setup a telescope to expand the beam size by 2-5 times.
\item Repeat the knife edge process.
\item Plot the results of both sets of measurements on paper and in excel.
\end{enumerate}
\end{column}
\end{columns}
\end{frame}

\begin{frame}\frametitle{Python Excursion}

\begin{center}
\includegraphics[width=6cm]{fig/python1.png}
\end{center}

\end{frame}

\section{Optical Tweezers}

\begin{frame}\frametitle{Microscope}
\begin{center}
\includegraphics[width=8cm]{fig/microscope1.jpg}
\end{center}

\[ m = m_o \cdot m_e = (-\frac{d_i}{d_o}) \cdot (-\frac{d_i^\prime}{d_o^\prime}) \]

\end{frame}

\begin{frame}\frametitle{Optical Tweezers -  Collimated Laser}
\begin{columns}
\begin{column}{7cm}
\begin{itemize}
\item Ashkin's award was being given for his ability to "realize an old dream of science fiction – using the radiation pressure of light to move physical objects" aka the Tractor Beam.
\item In 1987 Ashkin and his team were able to send a laser through a microscope's objective lens and trapping particles varying in size from a tens of nanometres up to tens of micrometres
\end{itemize}
\end{column}
\begin{column}{4cm}
\begin{center}
\includegraphics[width=4cm]{fig/nobel2018.jpg}
\end{center}
\end{column}
\end{columns}
\end{frame}

\begin{frame}\frametitle{Optical Tweezers -  Collimated Laser}
\begin{columns}
\begin{column}{5cm}
\begin{center}
\includegraphics[width=4cm]{fig/otg1.png}
\end{center}
\end{column}
\begin{column}{6cm}
When the bead is displaced from the beam center (right image), the larger momentum change of the more intense rays cause a net force to be applied back toward the center of the laser. When the bead is laterally centered on the beam (left image), the resulting lateral force is zero. But an unfocused laser still causes a force pointing away from the laser.
\end{column}
\end{columns}
\end{frame}

\begin{frame}\frametitle{Optical Tweezers - Focused Laser}
\begin{columns}
\begin{column}{5cm}
\begin{center}
\includegraphics[width=4cm]{fig/otg2.png}
\end{center}
\end{column}
\begin{column}{6cm}
In addition to keeping the bead in the center of the laser, a focused laser also keeps the bead in a fixed axial position: The momentum change of the focused rays causes a force towards the laser focus, both when the bead is in front (left image) or behind (right image) the laser focus. So, the bead will stay slightly behind the focus, where this force compensates the scattering force.
\end{column}
\end{columns}
\end{frame}

\begin{frame}\frametitle{Assignment: Optical Tweezers}
\begin{columns}
\begin{column}{4cm}
\begin{center}
\includegraphics[width=4cm]{fig/ot3.jpg}
\end{center}
\end{column}
\begin{column}{7cm}
Over the next few days, we will take a few hours at a time to work on EDU-OT3
\begin{itemize}
\item Carefully get the parts for the Optical Tweezers from the cabinets
\item Please don't write in the lab manuals, document progress in your notebooks.
\item Follow assembly instructions
\begin{itemize}
\item Take your time, alignment is critical
\item Take notes on the handouts on any documentation changes you feel are needed
\end{itemize}
\item Once constructed, follow the experiment instructions.
\end{itemize}
\end{column}
\end{columns}
\end{frame}



\section{Mirrors}

\begin{frame}\frametitle{Flat Mirror}

\begin{center}
\includegraphics[width=10cm]{fig/mirrorimage1.jpg}
\end{center}

\end{frame}

\begin{frame}\frametitle{Concave Spherical Mirrors - Thin Lens Equivalent}

\begin{center}
\includegraphics[width=10cm]{fig/mirrorimage2.jpg}
\end{center}

For a mirror that is large compared to the radius of curvature, the reflected rays do not cross at the same point. A parabolic mirror, the rays would indeed cross at a single point. However, parabolic mirrors are expensive. So, using a mirror that is small compared to the radius of curvature, leads to a well-defined focal point $F$, with $f = \frac{R}{2}$.

\end{frame}

\begin{frame}\frametitle{Convex Mirrors}

\begin{center}
\includegraphics[width=5cm]{fig/mirrorimage3.jpg}
\end{center}

\end{frame}

\begin{frame}\frametitle{Image Formation - Concave Mirrors}

\begin{center}
\includegraphics[width=8cm]{fig/mirrorimage4.jpg}
\end{center}

\end{frame}

\begin{frame}\frametitle{Image Formation - Concave Mirrors}

\begin{center}
\includegraphics[width=8cm]{fig/mirrorimage5.jpg}
\end{center}

\end{frame}

\begin{frame}\frametitle{Image Formation - Convex Mirrors}

\begin{center}
\includegraphics[width=8cm]{fig/mirrorimage6.jpg}
\end{center}

\end{frame}

\section{Vision}

\begin{frame}\frametitle{The Eye}

\begin{center}
\includegraphics[width=6cm]{fig/theI.jpg}
\end{center}

\end{frame}

\begin{frame}\frametitle{Rods and Cones and Color}

\begin{center}
\includegraphics[width=6cm]{fig/rodscones.jpg}
\end{center}

\end{frame}

\begin{frame}\frametitle{Visible Spectrum}

\begin{center}
\includegraphics[width=10cm]{fig/visSpec.jpg}
\end{center}

\end{frame}

\begin{frame}\frametitle{Spectrum of Light Sources}

\begin{center}
\includegraphics[width=10cm]{fig/lightSpec.jpg}
\end{center}

\end{frame}

\begin{frame}\frametitle{Chromatic Aberrations}

\begin{center}
\includegraphics[width=10cm]{fig/chromeAbb.jpg}
\end{center}

\end{frame}

\begin{frame}\frametitle{Correcting Chromatic Aberrations}

\begin{center}
\includegraphics[width=10cm]{fig/chromeAbb2.jpg}
\end{center}

\end{frame}

\begin{frame}\frametitle{Coma - Off Axis Abberation}

\begin{center}
\includegraphics[width=10cm]{fig/coma.jpg}
\end{center}

\end{frame}

\begin{frame}\frametitle{Spherical Aberrations}

\begin{center}
\includegraphics[width=10cm]{fig/sphereAbb.jpg}
\end{center}

\end{frame}




\section{Wave Optics}

\begin{frame}\frametitle{Light as an Electromagnetic Wave}
\begin{columns}
\begin{column}{7cm}
\begin{center}
\includegraphics[width=7cm]{fig/emwave2.png}
\end{center}
\end{column}
\begin{column}{4cm}
\begin{itemize}
\item Wavelength ($\lambda$) is the distance between peaks.
\item Amplitude (A) is the value between 0 and peak. Or half the distance between peak and trough.
\end{itemize}
\end{column}
\end{columns}
\end{frame}


\begin{frame}\frametitle{Electro-Magnetic Spectrum}
\begin{center}
\includegraphics[width=10cm]{fig/emspectrum.png}
\end{center}
\end{frame}


\begin{frame}\frametitle{Light as Wave (Wave Equation)}
\begin{columns}
\begin{column}{6cm}
\begin{center}
\includegraphics[width=5cm]{fig/emwave3.png}

\vspace{1cm}

\includegraphics[width=2cm]{fig/phi.png}

\end{center}
\end{column}
\begin{column}{6cm}
\begin{itemize}
\item Period (T) - time between peaks
\item Frequency (f) - oscillations per second
\item Basic wave equation
\[ c \cdot T = \lambda = \frac{c}{f}\]
\item Waves in time
\[y(t) = A \cdot \sin{(2 \pi f t+ \phi)}\]
\end{itemize}

\end{column}
\end{columns}
\end{frame}

\begin{frame}\frametitle{Documentation/Lab Notes}
\begin{columns}
\begin{column}{5cm}
\begin{center}
\includegraphics[height=6cm]{fig/labnotes.png}
\end{center}
\end{column}
\begin{column}{6cm}
Regular Feedback on Employers: Documentation, Documentation, Documentation
\begin{itemize}
\item Use the labnotes.doc (template on Teams) or hardcopy to record lab experiments.
\item List hypothesis or goal with experiment
\item Can reference manual, where applicable, for procedures
\item Reflections and Future Explanations are important steps, don't skip
\end{itemize}
\end{column}
\end{columns}
\end{frame}

\begin{frame}\frametitle{Assignment: Width of Human Hair}
\begin{columns}
\begin{column}{4cm}
\begin{center}
\includegraphics[width=4cm]{fig/waterloo.jpg}

\vspace{.25cm}

\includegraphics[width=4cm]{fig/hair.jpg}
\end{center}
\end{column}
\begin{column}{7cm}
\begin{itemize}
\item Mount a hair and shine laser (of wavelength $\lambda$ through it at a screen
\item Measure distance to the screen ($d$)
\item Measure the distance between dark spots ($s$)
\item Calculate the width of the hair ($w$)

\[ w = \frac{2 d \lambda}{s} \]

\item Try someone else's hair or a different color laser.
\end{itemize}
\end{column}
\end{columns}
\end{frame}


\begin{frame}\frametitle{Corpuscular or Waves}
\begin{columns}
\begin{column}{5cm}
In 1675, Sir Isaac Newton hypothesized that light was made up corpuscules (small particles) with the size/mass of the corresponding to different colors. 
\begin{center}
\includegraphics[width=2cm]{fig/corpReflect.png}
\includegraphics[width=2cm]{fig/corpRefract.png}
\end{center}

\begin{center}
\includegraphics[width=3cm]{fig/corpDoubleSlit.png}
\end{center}
\end{column}
\begin{column}{6cm}
In 1678, Christian Huygens, in order to explain the diffraction of light, proposed that every point on a wavefront is a source of wavelets that spread out in the forward direction at the same speed as the wave itself. The new wavefront is a line tangent to all of the wavelets.
\begin{center}
\includegraphics[width=5cm]{fig/huygens3.jpg}
\end{center}
\end{column}
\end{columns}
\end{frame}

\begin{frame}\frametitle{On the theory of light and color}

\begin{center}
\includegraphics[width=3cm]{fig/youngDoubleSlit.jpg}
\end{center}

Thomas Young, who had read Newton’s Opticks at age 17 saw some problems with Newton’s corpuscular theory.
\begin{itemize}
\item At interfaces (air/water) some light is reflected and some is refracted.
\item Different colors refracted to different degrees
\end{itemize} 

Observing sound (a compression wave in air):
\begin{itemize}
\item When two waves of sound cross, they interfere with each other
\item Light might exhibit interference phenomena as well
\item Developed basic idea for the double-slit experiment, which demonstrated the interference of light waves.
\end{itemize}
\end{frame}

\begin{comment}
\begin{frame}\frametitle{Maxwell's Equations: Electric $\vec{E}$ and Magnetic $\vec{B}$ Fields}
In 1864, James Clerk Maxwell predicted electromagnetic waves
\begin{itemize}
\item Gauss's Law: $\nabla \cdot \vec{E} = \frac{\rho}{\epsilon_0}$, where ${\rho}$ is enclosed charge
\item Guass's Law for Magnets: $\nabla \cdot \vec{B} = 0$
\item Farday's Law: $\nabla \times \vec{E} = -\frac{\partial \vec{B}}{\partial t}$
\item Ampere's Law: $\nabla \times \vec{B} = \mu_0 \vec{J} + \mu_0 \epsilon_0 \frac{\partial \vec{E}}{\partial t}$
\end{itemize}

where \newline
$\mu_0 = 4 \pi*10^{-7} \frac{F}{m}$ and 
$\epsilon_0 = 8.85*10^{-12} \frac{N m^2}{C}$ \newline

Maxwell noted that the speed of the electromagnetic wave is equal to the speed of light:


$\frac{1}{\sqrt{\mu_0 \epsilon_0}}= \frac{1}{\sqrt{4 \pi*10^{-7} \cdot 8.85*10^{-12}}} = 2.99*10^{8} \frac{m}{s} = c$
\end{frame}

\begin{frame}\frametitle{Huygens's Principle: Diffraction}
\begin{columns}
\begin{column}{4.5cm}
Every point on a wavefront is a source of wavelets that spread out in the forward direction at the same speed as the wave itself. The new wavefront is a line tangent to all of the wavelets.
\end{column}
\begin{column}{3.25cm}
\begin{center}
\includegraphics[width=2.5cm]{fig/huygens.jpg}
\end{center}
\end{column}
\begin{column}{3.25cm}
\begin{center}
\includegraphics[width=3cm]{fig/huygens.png}
\end{center}
\end{column}
\end{columns}
\end{frame}
\end{comment}






\begin{frame}\frametitle{Diffraction}
\begin{columns}
\begin{column}{6cm}
While the ray optics method is useful, if light indeed traveled in straight rays then there would be a pitch black shadow where the light is blocked by the wall.
\end{column}
\begin{column}{5cm}
\begin{center}
\includegraphics[width=5cm]{fig/doorway.jpg}
\end{center}
\end{column}
\end{columns}
\end{frame}

\begin{frame}\frametitle{Diffraction}
\begin{center}
\includegraphics[width=8cm]{fig/diffraction.jpg}
\end{center}

If we pass light through smaller openings, often called slits, we can use Huygens’s principle to see that light bends. The bending of a wave around the edges of an opening or an obstacle is called diffraction. Diffraction is a wave characteristic and occurs for all types of waves.
\end{frame}

\begin{frame}\frametitle{Interference}
\url{https://www.desmos.com/calculator}

\begin{columns}
\begin{column}{4cm}
\begin{center}
\includegraphics[width=3cm]{fig/desmos3.png}
\end{center}
\end{column}
\begin{column}{7cm}
\begin{center}
Constructive

\includegraphics[width=5cm]{fig/desmos4.png}

Destructive

\includegraphics[width=5cm]{fig/desmos5.png}
\end{center}
\end{column}
\end{columns}
\end{frame}

\begin{comment}


\begin{frame}\frametitle{Constructive Interference}
\url{https://www.desmos.com/calculator}
\begin{center}
\includegraphics[width=10cm]{fig/constructive.png}
\end{center}


\end{frame}


\begin{frame}\frametitle{Destructive Interference}
\url{https://www.desmos.com/calculator}
\begin{center}
\includegraphics[width=10cm]{fig/destructive.png}
\end{center}
\end{frame}

\begin{frame}\frametitle{Double Slit}
\begin{center}
\includegraphics[width=6cm]{fig/ds1.jpg}
\includegraphics[width=6cm]{fig/ds2.jpg}
\end{center}

\end{frame}
\end{comment}

\begin{frame}\frametitle{Path Length Difference}
\begin{columns}
\begin{column}{7cm}
\begin{itemize}
\item Constructive Interference - peaks are in phase
\[d \sin{(\theta}) = m \lambda, m = 0,1,-1,2,-2,....\]
\item Destructive Interference - peaks are perfectly out of phase
\[d \sin{(\theta}) = (m+\frac{1}{2})  \lambda, m = 0,1,-1,2,-2,....\]
\end{itemize}
\end{column}
\begin{column}{4cm}
\begin{center}
\includegraphics[width=4cm]{fig/ds3.jpg}
\end{center}
\end{column}
\end{columns}
\end{frame}

\begin{frame}\frametitle{Double Slit - Constructive Interference}
\begin{center}
\includegraphics[width=5cm]{fig/ds4.jpg}
\end{center}
\[ \tan{(\theta_1)} = \frac{y_1}{x} \]

Using small angle assumption\footnote{for small angles ($\theta$) measured in radians, $tan(\theta) = sin(\theta) = \theta$.}:
\[d y_1 = x \lambda \text{ for } m=1\]
\[d = \frac{x}{y_1} \lambda \]

\end{frame}

\begin{frame}\frametitle{Assignment: Double Slit}

\begin{center}
\includegraphics[width=8cm]{fig/thorDouble.jpg}
\end{center}

\vspace{1cm}

\begin{itemize}
\item Calculate the slit distance and slit width for three different double slits. 
\end{itemize}

\end{frame}



\begin{frame}\frametitle{Single Slit}
\begin{columns}
\begin{column}{5cm}
\begin{center}
\includegraphics[height=7cm]{fig/sl2.jpg}
\end{center}


\end{column}
\begin{column}{6cm}
\begin{center}
\includegraphics[width=3cm]{fig/sl1.jpg}

\includegraphics[width=3cm]{fig/sl3.jpg}
\end{center}
\[D = \frac{2 x \lambda}{y_{dark}}\]
\end{column}
\end{columns}
Double slit is actually the double slit result times the single slit.
\end{frame}

\begin{frame}\frametitle{Assignment: Single Slits and Pinholes}

\begin{center}
\includegraphics[width=8cm]{fig/thorDouble.jpg}
\end{center}

\vspace{1cm}

\begin{itemize}
\item Calculate the size of the various slits and pinholes using the distance between dark bands 
\end{itemize}

\end{frame}

\begin{frame}\frametitle{Diffraction Grating}
\begin{columns}
\begin{column}{5cm}
If light is passed through a large number of evenly spaced parallel slits, called a diffraction grating, the interference pattern is created that is very similar to the one formed by a double slit. \newline

Diffraction gratings can be made to work with the transmission or reflection of light. 
\end{column}
\begin{column}{6cm}
\begin{center}
\includegraphics[width=5.8cm]{fig/dg1.jpg}
\end{center}
\end{column}
\end{columns}
\end{frame}


\begin{frame}\frametitle{Diffraction Grating - Constructive Interference}
\begin{columns}
\begin{column}{7cm}
Similar to a double slit, the constructive interference happens on integral number of wavelengths:

\[d \cdot sin(\theta) = m \lambda \]

for, $m = 0, 1 , -1, 2, -2, ...$


\end{column}
\begin{column}{4cm}
\begin{center}
\includegraphics[width=3.8cm]{fig/dg2.jpg}
\end{center}
\end{column}
\end{columns}
\end{frame}

\begin{frame}\frametitle{Diffraction Grating - Spread of Wavelength}
\begin{columns}
\begin{column}{7cm}
We can find $\theta_R$, $y_R$, $\theta_V$, and $y_V$ by using:

\[d \cdot sin(\theta) = m \lambda \]

for, $m = 1$.


\end{column}
\begin{column}{4cm}
\begin{center}
\includegraphics[width=3.8cm]{fig/dg3.jpg}
\end{center}
\end{column}
\end{columns}
\end{frame}

\begin{frame}\frametitle{Assignment: Diffraction Grating}
\begin{columns}
\begin{column}{3cm}

\begin{center}
\includegraphics[width=3cm]{fig/diffslide.png}
\end{center}
\end{column}
\begin{column}{7.5cm}
\begin{itemize}
\item Setup: Red Laser, Diffraction Slide, Mount, Screen
\item Place diffraction slide approximately 3 inches from screen
\item Calculate wavelength of laser using diffraction equation
\item Repeat for Green and Blue/UV laser
\end{itemize}
\end{column}
\end{columns}
\end{frame}


\begin{frame}\frametitle{Black Body Radiation}
\begin{columns}
\begin{column}{7cm}
The EM spectrum radiated by a hot solid is linked directly to the solid’s temperature. An ideal radiator is one that has an emissivity of 1 at all wavelengths and, thus, is jet black. Ideal radiators are therefore called blackbodies, and their EM radiation is called blackbody radiation.
\end{column}
\begin{column}{4.5cm}
\begin{center}
\includegraphics[width=3cm]{fig/hephaestus.jpg}
\end{center}
\end{column}
\end{columns}

\begin{center}
\includegraphics[width=10cm]{fig/blacksmith.png}
\end{center}
\end{frame}


\begin{frame}\frametitle{Ultraviolet Catastrophe}
\begin{columns}
\begin{column}{7.2cm}

The ultraviolet catastrophe was the prediction of classical physics that an ideal black body at thermal equilibrium would emit an unbounded quantity of energy as wavelength decreased into the ultraviolet range.

\[ B_{f}(T) = \frac{2 f^2 k_B T}{c^2}, f = \frac{c}{\lambda}\]

where $B_f(T)$ is radiance or intensity at a given frequency

\[B_{f}(T) \rightarrow \infty, \text{ as } f \rightarrow \infty\]

\end{column}
\begin{column}{4cm}
\begin{center}
\includegraphics[width=4cm]{fig/blackbody.jpg}
\end{center}
\end{column}
\end{columns}
\end{frame}

\begin{frame}\frametitle{Quanta}

Max Planck assumed that electromagnetic radiation can be emitted or absorbed only in discrete packets, call quanta\footnote{Planck considered this a mathematical trick, not reality}, of energy:

\[ E_{quanta} = h f = h \frac{c}{\lambda}\]
where
\begin{itemize}
\item $h$ is Planck's Constant
\item $f$ is the frequency of light
\item $c$ is the speed of light
\item $\lambda$ is the wavelength of light
\end{itemize}

Applying the quantification to statistical mechanics:

\[ B_{\lambda}(\lambda,T) = \frac{2hc^2}{\lambda^5}\frac{1}{e^{(\frac{hc}{\lambda k_B T})}-1}   \]

\end{frame}



\begin{frame}\frametitle{Electron Energy}
\begin{columns}
\begin{column}{4cm}
\begin{center}
\includegraphics[width=4cm]{fig/vjc.jpg}
\end{center}

\vspace{.25cm}

\[ E_{quanta} = h f = h \frac{c}{\lambda}\]

\end{column}
\begin{column}{7cm}
\begin{itemize}
\item Charge of an electron ($-q$):
\[ q = 1.602 \cdot 10^{-19}\]
\item Energy of light or energy loss of electron that has passed through the LED potential
\[ E_{quanta} = V \cdot q\] 
\item Plank's constant
\[ h = E_{quanta}(\frac{\lambda}{c}) = \frac{V q \lambda}{c} \]
where $c = 3 \cdot 10^{8} (m/s)$
\end{itemize}

\end{column}
\end{columns}
\end{frame}




\begin{frame}\frametitle{Assignment: Plank and Colors}
\begin{columns}
\begin{column}{5cm}
\begin{center}
\includegraphics[width=5cm]{fig/plank.png}
\end{center}
\end{column}
\begin{column}{6cm}
Using "Determine the Wavelength of an LED" lab sheet:
\begin{itemize}
\item Build the circuit to measure voltage across Red LED
\item Also use Diffraction slide and ruler to calculate wavelength
\item Repeat for Yellow, Green, and Blue LEDs
\item Plot voltage vs wavelength in your notebook,excel, or python
\end{itemize}
\end{column}
\end{columns}
\end{frame}


\begin{frame}\frametitle{The story of the Blue LED}
\begin{center}
\includegraphics[width=8cm]{fig/blueLED.jpg}
\url{https://youtu.be/AF8d72mA41M?si=dGNIvEEUrIV7rAS0}
\end{center}
\end{frame}


\begin{frame}\frametitle{Assignment: Spectrometer}
\begin{center}
\includegraphics[width=9cm]{fig/spectrometer.jpg}
\end{center}
\begin{columns}
\begin{column}{5cm}
A spectrometer:
\begin{itemize}
\item Fiber input and aperture
\item Columnating mirror
\item Diffraction grating
\item Focusing mirror
\item CCD detector
\end{itemize}

\end{column}
\begin{column}{6cm}
Assignment:
\begin{itemize}
\item Measure RED LED wavelength with spectrometer
\item Compare to your values from Plank assignment
\item Repeat for other colors.
\end{itemize}
\end{column}
\end{columns}
\end{frame}


\begin{frame}\frametitle{Limits of Resolution}

Diffraction affects the detail that can be observed when light passes through an aperture.

\vspace{0.5cm}

\begin{center}
\includegraphics[width=10cm]{fig/rayleigh1.jpg}
\end{center}

\end{frame}


\begin{frame}\frametitle{Rayleigh Criterion}


Two points are just resolved if they are separated by an angle of $\theta = 1.22 \frac{\lambda}{D}$

\vspace{0.25cm}

\begin{center}
\includegraphics[width=8cm]{fig/rayleigh2.jpg}
\end{center}
\end{frame}

\begin{frame}\frametitle{Resolving Power}
\begin{columns}
\begin{column}{7cm}
The resolving power of a system is the smallest distance of separation ($x$) where two objects can be seen as distinct. It is given by the Rayleigh Criterion:

\[\theta = 1.22 \cdot \frac{\lambda}{D} = \frac{x}{d}\]
where $d$ is the between the lens objective and the object.

\end{column}
\begin{column}{4cm}
\begin{center}
\includegraphics[width=3cm]{fig/rayleigh3.jpg}
\end{center}
\end{column}
\end{columns}
\end{frame}

\begin{frame}\frametitle{Numerical Aperture (NA)}
\begin{columns}
\begin{column}{8cm}
The Numerical Aperture ($NA$) is the maximum acceptance angle of a fiber or lens. The $NA$ is a measure of the ability to gather light and resolve detail.

Using the $\alpha = \frac{\theta}{2}$ and the small angle approximation:

\[ sin(\alpha) = \frac{D/2}{d} = \frac{D}{2d} \]

The $NA$ is defined as $NA = n \cdot sin{\alpha}$ where $n$ is the index of refraction of the medium between the objective and point P.

\[ x = 1.22\frac{\lambda d}{D} = 1.22 \frac{\lambda}{2 sin(\alpha)} = 0.61 \frac{\lambda n}{NA}\]

\end{column}
\begin{column}{3cm}
\begin{center}
\includegraphics[width=3cm]{fig/rayleigh3.jpg}
\end{center}
\end{column}

\end{columns}

\footnote{Small angle approximation: $\sin(\theta) = \theta$ where $\theta$ is measured in radians.}

\end{frame}

\begin{frame}\frametitle{Huygens's Mirror}
\begin{columns}
\begin{column}{6cm}
A mirror reflects an incoming wave at an angle equal to the incident angle, verifying the law of reflection. As the wavefront strikes the mirror, wavelets are first emitted from the left part of the mirror and then the right. The wavelets closer to the left have had time to travel farther, producing a wavefront traveling in the direction shown.
\end{column}
\begin{column}{5cm}
\begin{center}
\includegraphics[width=5cm]{fig/huyMirror.jpg}
\end{center}
\end{column}
\end{columns}
\end{frame}

\begin{frame}\frametitle{Speed of Light in a Medium}
The speed of a wave is the frequency ($\frac{1}{s}$) times the wavelength ($m$):
\begin{equation}
c = f \lambda
\end{equation}
where in a vacuum, $c = 2.99 * 10^8 (\frac{m}{s})$. \newline

Light has wave characteristics in a medium other than a vacuum, as well. In this case, the speed and wavelength change, but the frequency stays the same. The speed of light in a medium is governed by its index of refraction ($n$), where $v = \frac{c}{n}$. \newline

Divide both sides of the above equation by $n$ yields:
\begin{equation}
v = \frac{c}{n} = \frac{f \lambda}{n} = f \lambda_n
\end{equation}
where $\lambda_n$ is the wavelength in the medium.
\end{frame}

\begin{frame}\frametitle{Huygens's Refraction}
\begin{columns}
\begin{column}{6cm}
Each wavelet to the right was emitted when the wavefront crossed the interface between the media. Since the speed of light is slower in the second medium, the waves do not travel as far in a given time, and the new wavefront changes direction as shown. This explains why a ray changes direction to become closer to the perpendicular when light slows down and can be used to derive Snell's Law.
\end{column}
\begin{column}{5cm}
\begin{center}
\includegraphics[width=5cm]{fig/huyRefract.jpg}
\end{center}
\end{column}
\end{columns}
\end{frame}

\section{Interferometry}

\begin{frame}\frametitle{Optical Component: Beam Splitter}

\begin{columns}
\begin{column}{4cm}
\begin{center}
\includegraphics[width=3cm]{fig/BS013.jpg}

\vspace{0.5cm}

\includegraphics[width=4cm]{fig/beamsplitter.png}


\end{center}
\end{column}
\begin{column}{7cm}
\begin{enumerate}
\item Beamsplitter allows half of the light to pass through the diagonal interface and half to reflect
\item Light can enter at any of the four faces and 1/2 reflected, 1/2 transmissed.
\item The arrows indicate the path where the specs are guarenteed.
\end{enumerate}
\end{column}
\end{columns}

\end{frame}



\begin{frame}\frametitle{Michelson Interferometer}

Invented by the American physicist Albert Abraham Michelson in 1887.

\begin{columns}
\begin{column}{7cm}
\begin{center}
\includegraphics[width=7cm]{fig/michelson2.jpg}
\end{center}
\end{column}
\begin{column}{3cm}
\begin{enumerate}
\item Laser
\item Beam Splitter
\item Mirror
\item Screen
\end{enumerate}
\end{column}
\end{columns}

\vspace{0.25cm}

\begin{itemize}
\item Laser light is divided by the beamsplitter, the partial beams are reflected by the mirrors and overlap again at the beamsplitter.
\item The light intensity on the screen is dependent on the path length difference ($\Delta s$) between the two paths $s_1$ and $s_2$. 
\end{itemize}
\end{frame}

\begin{frame}\frametitle{Interferometer Math}

\begin{center}
\includegraphics[width=6cm]{fig/interferometer2.jpg}
\end{center}


\begin{itemize}
\item Path 1 ($s_1$): Laser reflects off of Beamsplitter to Mirror 1 through Beamsplitter to Detector
\item Path 2 ($s_2$): Laser through of Beamsplitter to Mirror 2 reflects off Beamsplitter to Detector
\item $\Delta s = s_1 - s_2$: Constructive:$\Delta s = m \lambda$, Destructive:$\Delta s = (m+\frac{1}{2}) \lambda$
\end{itemize}
\end{frame}



\begin{frame}\frametitle{Interferometer Math - More Detailed}
\begin{itemize}
\item The Electric Field ($E_i$) is given by
\begin{equation}
\vert E_i \vert = \sqrt{R \cdot T} \cos{(\omega t + \phi_i)}
\end{equation}
where $T$ is the transmission capacity of the beamsplitter, $R$ is the relection capacity, and $\phi_i$ is the phase which value is defined by the actual optical path.
\item Intensity ($I$) on the screen is given by
\begin{equation}
I = c \epsilon_0 \vert E_1 + E_2 \vert^2
\end{equation}
\item If we assume that the transmission and reflection capacity are 0.5 then the average intensity ($\bar{I}$) is given by
\begin{equation}
\bar{I} = \frac{1}{4} c \epsilon_0 E_0^2 (1 + \cos{(\Delta \phi))}
\end{equation}
where $\Delta \phi = \frac{2 \pi}{\lambda} \Delta s$ and $\lambda$ is the wavelength
\end{itemize}
\end{frame}

\begin{frame}\frametitle{Interferometer Math - what does it mean}
Compare the centers:

\begin{center}
\includegraphics[width=10cm]{fig/fringe2.jpeg}
\end{center}

Why are there alternating concentric circles?
\end{frame}


\begin{frame}\frametitle{Assignment: Michelson Interferometer}
\begin{columns}
\begin{column}{5cm}
\begin{center}
\includegraphics[width=4.5cm]{fig/mint2.jpg}
\end{center}
\end{column}
\begin{column}{6cm}
Using the ThorLabs Michelson Interferometer Kit:
\begin{itemize}
\item Setup per Chapter 4
\item Complete Preliminary Test (Chapter 6.1)
\item Determine Laser Wavelenth (Chapter 6.2), document using the Labnotes worksheet.
\item Draw a diagram of the Laser Wavelength setup.
\item When told: teardown setup, but keep optics in mounts
\end{itemize}
\end{column}
\end{columns}
\end{frame}

\begin{frame}\frametitle{Assignment: Michelson Interferometer 2}
\begin{columns}
\begin{column}{5cm}
\begin{center}
\includegraphics[width=4cm]{fig/michelson3.jpg}
\end{center}
\end{column}
\begin{column}{6cm}
Using the ThorLabs Michelson Interferometer Kit and documenting your experiment on the Labnotes worksheet:
\begin{itemize}
\item Setup per Chapter 4
\item Use as spectrometer (Chapter 6.3)
\item Explore coherence (Chapter 6.4)
\item When told: teardown setup, but keep optics in mounts
\end{itemize}
\end{column}
\end{columns}

\vspace{1.0cm}

Optional: if instructed, repeat Chapter 6.3 Spectrometer experiment using the laser from the Optical Tweezers at above and below the lasing threshold.
\end{frame}

\begin{frame}\frametitle{Assignment: Michelson Interferometer 3}
\begin{columns}
\begin{column}{5cm}
\begin{center}
\includegraphics[width=4cm]{fig/michelson4.jpg}

\vspace{0.25cm}

\includegraphics[width=4cm]{fig/michelson5.jpg}
\end{center}
\end{column}
\begin{column}{7cm}
Using the ThorLabs Michelson Interferometer Kit and documenting your experiment on the Labnotes worksheet:
\begin{itemize}
\item Setup per Chapter 4
\item Measure index of refraction (Chapter 6.5)
\item Measure thermal expansion (Chapter 6.6)
\item When told: teardown setup completely, clean optics, and store
\item As a group, answer the questions in Chapter 8.
\end{itemize}
\end{column}
\end{columns}
\end{frame}

\section{Polarization}

\begin{frame}\frametitle{Electromagnetic Wave}

\begin{center}
\includegraphics[width=6cm]{fig/em_wave.png}
\end{center}

An Electromagnetic (EM) wave is a transverse wave where the electric and magnetic fields are perpendicular to each other and to the direction of propagation. 

\begin{itemize}
\item Light is called unpolarized if the direction of this electric field fluctuates randomly in time. 
\item If the direction of the electric field of light is well defined, it is called polarized light. 
\end{itemize}

\end{frame}



\begin{frame}\frametitle{Linear Polarization}

We define the direction of polarization to be the direction parallel to the electric field.

\begin{center}
\includegraphics[width=6cm]{fig/polarization1.png}
\end{center}

\end{frame}

\begin{frame}\frametitle{Linear Polarizer}

Natural light has polarizations in random directions, it is unpolarized.

\begin{center}
\includegraphics[width=6cm]{fig/polarizer.jpg}
\end{center}

\end{frame}


\begin{frame}\frametitle{Assignment: Malus' Polarization}
\begin{columns}
\begin{column}{4cm}
\begin{center}
\includegraphics[width=3cm]{fig/rsp1.jpg}
\end{center}
\end{column}
\begin{column}{7cm}
Documenting your work with the Labnotes worksheet:
\begin{itemize}
\item Mount a linear polarizer slide horizontally
\item Mount a cutout of the linear polarizer film to a RSP1 mount
\item Shine a laser through both polarizers on a power meter, setting the maximum power to a rotation of $0^{\circ}$
\item Take power measurements at every $15^{\circ}$ of rotation
\item Plot the results
\end{itemize}
\end{column}
\end{columns}
\end{frame}

\begin{frame}\frametitle{Malus's Law}
In 1810, French engineer Étienne-Louis Malus studied polarization after casually examining a doubly refracting prism of quartz the sunlight reflected from the windows of the Luxembourg palace.


\begin{center}
\includegraphics[width=8cm]{fig/malus.jpg}
\end{center}

\[ I = I_0 cos^2(\phi)\]

\end{frame}

\begin{frame}\frametitle{Rotating Linear Polarization with Half Wave Plate}
\begin{columns}
\begin{column}{5cm}
\begin{center}
\includegraphics[width=5cm]{fig/HWP.png}

\includegraphics[width=5cm]{fig/thorHWP.jpg}
\end{center}

\end{column}
\begin{column}{6cm}
\begin{itemize}
\item Half-wave plates rotate the axis of polarization for linearly polarized light
\item The plate is made of Birefringence material where the index of refraction is different along two perpendicular crystal axes
\item The thickness is set so that light polarizated along one axis is retarded a half-wavelength from the other axis. 
\end{itemize}

\end{column}
\end{columns}
\end{frame}



\begin{frame}\frametitle{Incoming $45^\circ$ with Half-Wave Plate}
From \url{https://emanim.szialab.org/}

\begin{center}
\includegraphics[width=10cm]{fig/waveplate_45_2.png}
\end{center}
\end{frame}

\begin{frame}\frametitle{Incoming $37^\circ$ with Half-Wave Plate}
From \url{https://emanim.szialab.org/}

\begin{center}
\includegraphics[width=10cm]{fig/waveplate_37_2.png}
\end{center}
\end{frame}

\begin{frame}\frametitle{Reflection vs Polarization}
\begin{columns}
\begin{column}{5cm}
\begin{center}
\includegraphics[width=5cm]{fig/svsp.png}
\end{center}
\end{column}
\begin{column}{6cm}
When light in incident on the interface of materials with different indices of refraction: 
\begin{itemize}
\item Polarized light that is parallel to the plane of incidence (P-Polarization) is refracted
\item Polarized light that is perpendicular (senkrecht in German) to the plan (S-Polarization) is transmitted.
\item The angle were the reflected and refracted angles are at $90^{\circ}$ to each other, the angle of incidence is called the Brewster's Angle.
\end{itemize}
\end{column}
\end{columns}
\end{frame}

\begin{frame}\frametitle{Polarizing Beam Splitter}
\begin{columns}
\begin{column}{5cm}
\begin{center}
\includegraphics[width=4cm]{fig/pbsEd.jpg}
\includegraphics[width=4cm]{fig/PBS252.png}
\end{center}
\end{column}
\begin{column}{6cm}
\begin{itemize}
\item Similar to an non-polarized beam splitter, it splits the light path.
\item P-polarization is transmitted, S-polarization is reflected by interface
\item By changing the polarization direction, the amount of light transmitted vs reflected can be adjusted.
\end{itemize}
\end{column}
\end{columns}
\end{frame}

\begin{frame}\frametitle{Assignment: More Polarization}
\begin{columns}
\begin{column}{2cm}
\begin{center}
\includegraphics[width=2.5cm]{fig/PBS252b.jpg}
\end{center}
\end{column}
\begin{column}{8.5cm}
Documenting your work with the Labnotes:
\begin{itemize}
\item Mount the half-wave plate on a rotating mount.
\item Shine the laser through the linear polarizer slide, the half-wave plate, the beamsplitter
\item Rotate the half-wave plate so all the light transmits through the beamsplitter
\item Measure the power with the power meter and then the detector connected to an oscilloscope. Find the scaling factor between volts and power.
\item Rotate the half-wave plate 45 degrees and measure the reflected beam power. (Don't move the detector attached to the o-scope
\item Rotate the half-wave plate recording both measurements. Convert between volts and power.
\end{itemize}
\end{column}
\end{columns}
\end{frame}



\begin{frame}\frametitle{Incoming $37^\circ$ with Quarter-Wave Plate}
From \url{https://emanim.szialab.org/}

\begin{center}
\includegraphics[width=10cm]{fig/waveplate_37_4.png}
\end{center}
\end{frame}

\begin{frame}\frametitle{Incoming $45^\circ$ with Quarter-Wave Plate}
From \url{https://emanim.szialab.org/}

\begin{center}
\includegraphics[width=10cm]{fig/waveplate_45_4.png}
\end{center}
\end{frame}

\begin{frame}\frametitle{Linear Polarization - Off Axis}

\url{https://www.mathsisfun.com/algebra/vector-calculator.html}

\begin{center}
\includegraphics[width=12cm]{fig/pol1.jpg}
\end{center}

\begin{equation}
\vec{E} = 3 \sin{(t)} \vec{i} + 4 \sin{(t)} \vec{j}
\end{equation}

\end{frame}




\begin{frame}\frametitle{Circular Polarization}

\url{https://www.mathsisfun.com/algebra/vector-calculator.html}


\begin{center}
\includegraphics[width=12cm]{fig/pol2.jpg}
\end{center}

X lags Y by $\frac{\pi}{2}$:
\begin{equation}
\vec{E} = 4 \sin{(t + \frac{\pi}{2})} \vec{i} + 4 \sin{(t)} \vec{j}
\end{equation}

\end{frame}

\section{Quantum Cryptography - Quantum Key Distribution}

\begin{frame}\frametitle{Cryptography}

Cryptography is the process of encoding information so that only intended recipient can read it. 

\vspace{0.25cm}

\begin{center}
\includegraphics[width=12cm]{fig/crypt1.png}
\end{center}

\vspace{0.25cm}

The art of cryptography has been used to code messages for thousands of years and continues to be used in bank cards, computer passwords, and ecommerce.
\end{frame}


\begin{frame}\frametitle{Cesear Cipher}
\begin{columns}
\begin{column}{4.5cm}
\begin{center}
\includegraphics[width=4.5cm]{fig/crypt4.png}
\end{center}
\end{column}
\begin{column}{7cm}
Caesar cipher or the shift cipher is one of the simplest and most widely known encryption techniques. It is a type of substitution cipher in which each letter in the plaintext is replaced by a letter some fixed number of positions down the alphabet. For example, with a left shift of 3, D would be replaced by A, E would become B, and so on. The method is named after Julius Caesar, who used it in his private correspondence.
\end{column}
\end{columns}
\end{frame}


\begin{frame}\frametitle{Vernam Cipher}
\begin{columns}
\begin{column}{7cm}
\begin{center}
\includegraphics[width=7cm]{fig/crypt5a.jpg}

\vspace{1cm}

\includegraphics[width=7cm]{fig/crypt5b.jpg}
\end{center}
\end{column}
\begin{column}{4.5cm}
Encrypt:
\begin{enumerate}
\item Take numerical value of each letter
\item Add numerical value of key
\item Subtract 26 if needed
\item Convert to letters
\end{enumerate}

\vspace{1cm}

Decrypt: Reverse process
\end{column}
\end{columns}
\end{frame}


\begin{frame}\frametitle{Symmetric Key Encryption}
\begin{center}
\includegraphics[width=6cm]{fig/crypt2.png}
\end{center}
\end{frame}

\begin{frame}\frametitle{Block Cipher (i.e., DES)}
\begin{columns}
\begin{column}{5cm}
\begin{center}
\includegraphics[width=5cm]{fig/crypt6.png}
\end{center}
\end{column}
\begin{column}{6.5cm}
\begin{itemize}
\item A block cipher takes a block of plaintext bits and generates a block of ciphertext bits, generally of same size. 
\item Data Encryption Standard (developed at IBM) was the standard from 1974 to 2005. 
\item DES uses a 56-bit key and found it is susceptible to decryption attacks.
\end{itemize}
\end{column}
\end{columns}
\end{frame}


\begin{frame}\frametitle{Advanced Encryption Standard}
AES is a symmetric key encryption method developed by a pair of Belgian cryptographers named Joan Daemen and Vincent Rijmen.

\vspace{1cm}

\begin{columns}
\begin{column}{5cm}
\begin{center}
\includegraphics[width=5cm]{fig/crypt7.jpg}
\end{center}
\end{column}
\begin{column}{7cm}
\begin{itemize}
\item AES supports 128-bit data blocks using 128, 192, or 256-bit keys. 
\item Substitutions, permutations, and mixing are used to secure encryption. 
\item The transmitter and receiver share a key for symmetric encryption. 
\end{itemize} 
\end{column}
\end{columns}
\end{frame}

\begin{frame}\frametitle{Asymmetric Key Encryption}
\begin{center}
\includegraphics[width=6cm]{fig/crypt3.png}
\end{center}
\end{frame}


\begin{frame}\frametitle{RSA Algorithm}
The RSA algorithm is named after Ron Rivest, Adi Shamir, and Len Adleman, who invented it in 1977

\begin{columns}
\begin{column}{5cm}
\begin{center}
\includegraphics[width=5cm]{fig/crypt8.jpg}
\end{center}
\end{column}
\begin{column}{7cm}
\begin{itemize}
\item RSA cryptosystem is the most wildly-used public key
\item It can encrypt messages with the need to exchange a secret key separately
\item The RSA algorithm can be used in both public key encryption and digital signatures
\item Its security is based on the difficulty of factoring large integers.
\end{itemize} 
\end{column}
\end{columns}
\end{frame}

\begin{frame}\frametitle{Shor's Algorithm}
Shor's algorithm is a quantum algorithm for finding the prime factors of an integer. It was developed in 1994 by the American mathematician Peter Shor.

\begin{center}
\includegraphics[width=6cm]{fig/crypt9.png}
\end{center}

\begin{itemize}
\item Classical Computer - factoring prime number is an exponential problem
\item Shor's Algorithm is a polynomial problem.
\end{itemize}
\end{frame}




\begin{frame}\frametitle{BB84 - Alice and Bob}
\begin{center}
\includegraphics[width=10cm]{fig/crypt10.png}
\end{center}

\end{frame}

\begin{frame}\frametitle{No Cloning Theorem}
\begin{columns}
\begin{column}{5cm}
\begin{center}
\includegraphics[width=4cm]{fig/crypt13.jpg}
\end{center}
\end{column}
\begin{column}{5cm}
The no-cloning theorem states that it is impossible to create an independent and identical copy of an arbitrary unknown quantum state
\end{column}
\end{columns}
\end{frame}



\begin{frame}\frametitle{BB84 - What About Eve}
\begin{center}
\includegraphics[width=10cm]{fig/crypt11.jpg}
\end{center}

\end{frame}



\begin{frame}\frametitle{Assignmet: Quantum Key Distribution}
\begin{columns}
\begin{column}{5cm}
\begin{center}
\includegraphics[width=5cm]{fig/ThorQKD.jpg}
\end{center}
\end{column}
\begin{column}{6.5cm}
\begin{itemize}
\item Assemble the Alice, Bob, and Eve units. Notice how Half-Wave Plates are used to select basis.
\item Using the Alice and Bob lesson sheets, create and send a code between Alice and Bob
\item Validate the same bits are received when the same basis are chosen.
\item Repeat using a new message, Eve eavesdropping and the Alice, Bob, and Eve worksheets
\item Compare basis and bits to see that Eve is detected
\end{itemize}
\end{column}
\end{columns}
\end{frame}

\section{The Atom and Quantum Mechanics}

\begin{frame}\frametitle{The Electron}
The English physicist, J.J. Thompson experimented with Cathode-Ray Tubes passing the "ray" through both electric and magnetic fields.
\begin{columns}
\begin{column}{7cm}


\begin{itemize}
\item E-field used to deflect beam:
\[ F = q_e E \]
\item The vertical deflection was given by:
\[ a = \frac{F}{m_e} = \frac{q_e E}{m_e} \]

\item Leading to:
\[ \frac{q_e}{m_e} = \frac{a}{E} = -1.76 \times 10^{11} (C/kg) \]
\end{itemize}
\end{column}
\begin{column}{4.5cm}
\begin{center}
\includegraphics[width=4cm]{fig/crt1.jpg}
\includegraphics[width=4cm]{fig/crt2.jpg}
\end{center}
\begin{itemize}
\item CRT w/Hydrogen Ion:
\[ \frac{q_p}{m_p} = 9.58 \times 10^7 (C/kg) \]
\end{itemize}

\end{column}
\end{columns}
\end{frame}


\begin{frame}\frametitle{The Electron}
American physicist, Robert Millikan improved on the experiment:
\begin{columns}
\begin{column}{7cm}


\begin{itemize}
\item Drop of oil in E-field
\[ m_{drop} g = q_e E\] 
\item where, the E-field created by voltage
\[ E = \frac{V}{d} \]
\end{itemize}
\end{column}
\begin{column}{4.5cm}
\begin{center}
\includegraphics[width=4cm]{fig/millkan.jpg}

\end{center}
\end{column}
\end{columns}
\begin{itemize}
\item This lead to the mass of an electron:
\[ q = \frac{m_{drop} g}{E} = \frac{m_{drop} gd}{V} = -1.6 \times 10^{-19} (C)\]
\end{itemize}
\end{frame}

\begin{frame}\frametitle{Mass of Electron and Proton}
With the charge of the electron known (and thus the proton), that allowed the mass of the electron and proton to be calculated:

\[ m_e = \frac{q_e}{(\frac{q_e}{m_e})} = \frac{-1.6 \times 10^{-19}{-1.76 \times 10^{-11}}}  = 9.11 \times 10^{-31} (kg) \]

\vspace{1cm}

And, the mass the proton:

\[ m_p = 1.67 \times 10^{-27} (kg) \]


\end{frame}

\begin{frame}\frametitle{The Nucleus}
Australian physicist, Lord Ernest Rutherford conducted scattering experiments where a thin gold foil was placed in a beam of alpha particles (a double charged helium atom) and the resulting scattering was observed. Based on scattering angles, Rutherford estimated the size of the nucleus to be $10^{-15} (m),$ or 100,000 smaller than the radius of an atom.
\begin{center}
\includegraphics[width=8cm]{fig/nucleus.jpg}
\end{center}
\end{frame}

\begin{frame}\frametitle{Models of the Atom}
\begin{center}
\includegraphics[width=10cm]{fig/atom1.jpg}
\end{center}
\end{frame}


\begin{frame}\frametitle{Black Body Radiation}
\begin{columns}
\begin{column}{7cm}
The EM spectrum radiated by a hot solid is linked directly to the solid’s temperature. An ideal radiator is one that has an emissivity of 1 at all wavelengths and, thus, is jet black. Ideal radiators are therefore called blackbodies, and their EM radiation is called blackbody radiation.
\end{column}
\begin{column}{4.5cm}
\begin{center}
\includegraphics[width=3cm]{fig/hephaestus.jpg}
\end{center}
\end{column}
\end{columns}

\begin{center}
\includegraphics[width=10cm]{fig/blacksmith.png}
\end{center}
\end{frame}


\begin{frame}\frametitle{Ultraviolet Catastrophe}
\begin{columns}
\begin{column}{7.2cm}

The ultraviolet catastrophe was the prediction of classical physics that an ideal black body at thermal equilibrium would emit an unbounded quantity of energy as wavelength decreased into the ultraviolet range.

\[ B_{\nu}(T) = \frac{2 \nu^2 k_B T}{c^2}, \nu = \frac{c}{\lambda}\]

\[B_{\nu}(T) \rightarrow \infty, \text{ as } \nu \rightarrow \infty\]

\end{column}
\begin{column}{4cm}
\begin{center}
\includegraphics[width=4cm]{fig/blackbody.jpg}
\end{center}
\end{column}
\end{columns}
\end{frame}

\begin{frame}\frametitle{Quanta}

Max Planck assumed that electromagnetic radiation can be emitted or absorbed only in discrete packets, call quanta\footnote{Planck considered this a mathematical trick, not reality}, of energy:

\[ E_{quanta} = h \nu = h \frac{c}{\lambda}\]
where
\begin{itemize}
\item $h$ is Planck's Constant
\item $\nu$ is the frequency of light
\item $c$ is the speed of light
\item $\lambda$ is the wavelength of light
\end{itemize}

Applying the quantification to statistical mechanics:

\[ B_{\lambda}(\lambda,T) = \frac{2hc^2}{\lambda^5}\frac{1}{e^{(\frac{hc}{\lambda k_B T})}-1}   \]

\end{frame}


\begin{frame}\frametitle{Photoelectric Effect}
\begin{columns}
\begin{column}{7.5cm}
Albert Einstein realized the photoelectric effect could be explained only if EM radiation is itself quantized $E = hf$
\begin{itemize}
\item There is a minimum energy ($f_0$) before any electrons are ejected.
\item Electrons are ejected without delay.
\item The number of electrons ejected is proportional to intensity of EM.
\item The maximum kinetic energy of the electrons is independent of intensity.
\item The maximum kinetic energy is given by $KE_e = hf - BE$. 
\end{itemize}
A quantum of light is called a photon.
\end{column}
\begin{column}{4cm}
\begin{center}
\includegraphics[width=3.5cm]{fig/photoelectric.png}
\includegraphics[width=3.5cm]{fig/photoelectric2.jpg}
\end{center}
\end{column}
\end{columns}
\end{frame}

\begin{frame}\frametitle{Bohr's Atom}
\begin{center}
\includegraphics[width=7cm]{fig/hydrogenspectrum.png}
\end{center}

The observed hydrogen-spectrum wavelengths can be calculated using the following formula:

\[\frac{1}{\lambda} = R (\frac{1}{n_f^2} - \frac{1}{n_i^2})\]
where the Rydberg constant (R): $R = 1.097 \times 10^7 (m^{-1})$.

\begin{itemize}
\item Lyman Series: $n_f = 1$.
\item Balmer Series: $n_f = 2$.
\item Paschen Series: $n_f = 3$.
\end{itemize}
\end{frame}

\begin{frame}\frametitle{Bohr's Atom}
\begin{center}
\includegraphics[width=8cm]{fig/lyman.png}
\end{center}
\end{frame}


\begin{frame}\frametitle{Bohr's Atom}
\begin{columns}
\begin{column}{7cm}
Neils Bohr, Dutch physicist, derived the spectrum from the planetary model and:
\begin{itemize}
\item Only certain orbits area allowed (orbits of electrons in atoms are quantized)
\item Each orbit has different energy: absorb to move to a higher orbit, drop to lower orbit by emitting.
\item Energy absorbed/emitted is also quantized, producing discrete spectra.
\item Energy to move from one orbit to another
\[ \Delta E = hf = E_i - E_f\]
\end{itemize}
\end{column}
\begin{column}{4.5cm}
\begin{center}
\includegraphics[width=4cm]{fig/bohr.jpg}
\end{center}
\end{column}
\end{columns}
\end{frame}

\section{The Laser}

\begin{frame}\frametitle{Stimulated Emission}
\begin{center}
\includegraphics[width=10cm]{fig/laser1.png}
\end{center}
\end{frame}

\begin{frame}\frametitle{Light Amplification $\rightarrow$ LASER}
\begin{center}
\includegraphics[width=10cm]{fig/laser2.png}
\end{center}
\end{frame}

\begin{frame}\frametitle{Coherence}
\begin{center}
\includegraphics[width=8cm]{fig/laser3.jpg}
\end{center}
\end{frame}

\begin{frame}\frametitle{HeNe Laser}
\begin{center}
\includegraphics[width=8cm]{fig/HeNe_transitions.jpg}
\end{center}
\end{frame}

\section{Entanglement}

\begin{frame}\frametitle{BBO Crystals}
\begin{center}
\includegraphics[width=8cm]{fig/BBOcrystal.jpg}
\end{center}
\end{frame}

\begin{frame}\frametitle{Spontaneous Parametric Down Conversion}
\begin{center}
\includegraphics[width=8cm]{fig/SPDC.png}
\end{center}
\end{frame}

\begin{frame}\frametitle{Spontaneous Parametric Down Conversion}
\begin{center}
\includegraphics[width=8cm]{fig/SPDC2.jpg}
\end{center}
\end{frame}


\begin{frame}\frametitle{Single Photon?}
Grangier-Roger-Aspect experiment

\begin{columns}
\begin{column}{7cm}
\begin{itemize}
\item Introduce beamsplitter
\item Single photon transmitted or reflected, but not both
\item Autocorrelations: \[g^{(2)}(0) = \frac{N_{TAB} \cdot N_T}{N_{TA} \cdot N_{TB}} \]
\begin{itemize}
\item Coherent (Laser): $g^{(2)}(0) = 1$
\item Classical Light: $g^{(2)}(0) > 1$
\item Non-Classical Light: $g^{(2)}(0) < 1$
\end{itemize}

\end{itemize}
\end{column}
\begin{column}{4.5cm}
\begin{center}
\includegraphics[width=4cm]{fig/grangier.jpg}
\end{center}
\end{column}
\end{columns}
\end{frame}

\section{Schrodenger's Atom}

\begin{frame}\frametitle{Waves and Quantization}
\begin{columns}
\begin{column}{6.5cm}
\begin{itemize}
\item Per de Broglie proposed matter has wave-like properties ($\lambda = \frac{h}{p}$)
\item Electrons can only exist in locations where they interfere constructively.
\item Not all orbits produce constructive interference, so not all allowed
\[ n \lambda_n = 2 \pi r_n \text{, for n = 1, 2, 3, ...} \]
\[ \frac{nh}{m_e v} = 2 \pi r_n \]
Angular Momentum (L):
\[ L = m_e v r_n = n \frac{h}{2 \pi}  \text{, for n = 1, 2, 3, ...} \]
\end{itemize}
\end{column}
\begin{column}{5cm}
\begin{center}
\includegraphics[width=5cm]{fig/bohr1.jpg}

\vspace{1cm}

\includegraphics[width=5cm]{fig/bohr2.jpg}
\end{center}
\end{column}
\end{columns}
\end{frame}

\begin{frame}\frametitle{Probability Cloud}
\begin{columns}
\begin{column}{6.5cm}
Due to the wave nature of matter, the idea of well-defined orbits gives way to a model in which there is a cloud of probability, consistent with the Heisenberg uncertainty principle. 
\end{column}
\begin{column}{5cm}
\begin{center}
\includegraphics[width=5cm]{fig/schrodenger1.jpg}
\end{center}
\end{column}
\end{columns}
\end{frame}

\begin{frame}\frametitle{Patterns in Spectra - Zeeman Effect}
\begin{columns}
\begin{column}{6.5cm}
Dutch physics student Pieter Zeeman investigated how spectra are affected by magnetic fields. In the presence of an external\footnote{Very percise measurements have shown that spectrail lines are doublets (split in two) apparently by the magnetic fields within the atom itself.} magnetic fields, the spectral lines split into two or more separate lines.
\end{column}
\begin{column}{5cm}
\begin{center}
\includegraphics[width=5cm]{fig/zeeman1.jpg}
\end{center}
\end{column}
\end{columns}
\end{frame}


\section{What is a Qubit}

\begin{frame} \frametitle{Bit vs Qubit}
\begin{center}
\includegraphics[width=11cm]{fig/bitQubit.png}
\end{center}

\end{frame}

\section{Quantum Computing}

\begin{frame}\frametitle{Types of Quantum Computers}
\begin{itemize}
\item Superconducting
\item Photonic
\item Neutral Atom
\item Trapped Ion
\item Quantum Dots
\item Diamond Nitrogen Vacancies
\end{itemize}
\end{frame}

\begin{frame}\frametitle{Quantum Computing: Superconducting}
One of the most popular types of quantum computers is a superconducting qubit quantum computer. Usually made from superconducting materials, these quantum computers utilize tiny electrical circuits to produce and manipulate qubits. When using superconducting qubits, gate operations can be performed quickly.

Companies actively researching and manufacturing superconducting quantum computers include Google, IBM, IQM and Rigetti Computing to name just a few.
\end{frame}


\begin{frame}\frametitle{Quantum Computing: Photonics}
These types of quantum computers use photons (particles of light) to carry and process quantum information. For large-scale quantum computers, photonic qubits are a promising alternative to trapped ions and neutral atoms that require cryogenic or laser cooling.
\end{frame}


\begin{frame}\frametitle{Quantum Computing: Neutral Atom}
Quantum computing based on neutral atoms involves atoms suspended in an ultrahigh vacuum by arrays of tightly focused laser beams called optical tweezers, though not all neutral atom companies use optical tweezers. Neutral atom quantum computers are less sensitive to stray electric fields, which makes them a good option for quantum processors.
\end{frame}


\begin{frame} \frametitle{Trapped Ions}
\begin{center}
\includegraphics[width=5cm]{fig/qscoutTrappedIon.jpg}
\end{center}

A trapped ion quantum computer involves using atoms or molecules with a net electrical charge known as “ions” that are trapped and manipulated using electric and magnetic fields to store and process quantum information. As trapped ions can be isolated from their environment, they are useful for precision measurements and other applications requiring high levels of stability and control. Also, the qubits can remain in a superposition state for a long time before becoming decoherent.

Representing the trapped ions community of companies in the quantum space, we have Quantinuum (a company that came out of the merger between Cambridge Quantum Computing and Honeywell Quantum Solutions), IonQ, Quantum Factory, Alpine Quantum Technologies, eleQtron amongst others

\end{frame}


\begin{frame}\frametitle{Quantum Computing: Quantum Dots}
A quantum dot quantum computer uses silicon qubits made up of pairs of quantum dots. In theory for quantum computers, such ‘coupled’ quantum dots could be used as robust quantum bits, or qubits.

Companies focused on this area include Diraq, Siquance and Quantum Motion.
\end{frame}

\begin{frame}\frametitle{Quantum Computing: NV Diamond}

\end{frame}



\end{document}
