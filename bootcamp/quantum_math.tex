\documentclass{beamer}
\setbeamertemplate{navigation symbols}{}
\usepackage{comment}

\setbeamercolor{frametitle}{fg=black,bg=white}
\setbeamercolor{title}{fg=black,bg=yellow!85!orange}
\usetheme{AnnArbor}

\usepackage{textpos} % package for the positioning
\usepackage{listings}
\usepackage{xcolor}
\usepackage[most]{tcolorbox}
\usepackage{mathtools}
\usepackage{graphicx}
\usepackage{graphbox}
\usepackage{caption}
\DeclareCaptionType{code}[Code Listing][List of Code Listings] 

\definecolor{codegreen}{rgb}{0,0.6,0}
\definecolor{codegray}{rgb}{0.5,0.5,0.5}
\definecolor{codepurple}{rgb}{0.58,0,0.82}
\definecolor{backcolour}{rgb}{0.95,0.95,0.92} 
\lstdefinestyle{mystyle}{
    backgroundcolor=\color{backcolour},   
    commentstyle=\color{codegreen},
    keywordstyle=\color{magenta},
    numberstyle=\tiny\color{codegray},
    stringstyle=\color{codepurple},
    basicstyle=\ttfamily\footnotesize,
    breakatwhitespace=false,         
    breaklines=true,                 
    captionpos=b,                    
    keepspaces=true,                 
    numbers=left,                    
    numbersep=5pt,                  
    showspaces=false,                
    showstringspaces=false,
    showtabs=false,                  
    tabsize=2
}

\lstset{style=mystyle}

\lstdefinelanguage
   [x64]{Assembler}     % add a "x64" dialect of Assembler
   [x86masm]{Assembler} % based on the "x86masm" dialect
   % with these extra keywords:
   {morekeywords={CDQE,CQO,CMPSQ,CMPXCHG16B,JRCXZ,LODSQ,MOVSXD, %
                  POPFQ,PUSHFQ,SCASQ,STOSQ,IRETQ,RDTSCP,SWAPGS, %
                  rax,rdx,rcx,rbx,rsi,rdi,rsp,rbp, %
                  r8,r8d,r8w,r8b,r9,r9d,r9w,r9b, %
                  r10,r10d,r10w,r10b,r11,r11d,r11w,r11b, %
                  r12,r12d,r12w,r12b,r13,r13d,r13w,r13b, %
                  r14,r14d,r14w,r14b,r15,r15d,r15w,r15b}} %


\beamersetuncovermixins{\opaqueness<1>{25}}{\opaqueness<2->{15}}

%Copyright
\addtobeamertemplate{frametitle}{}{%
\begin{textblock*}{50mm}(0cm,-1.25cm)
\color{yellow!85!orange}
\tiny{Copyright \copyright 2024 CNM.}
\end{textblock*}}

% position the logo
\addtobeamertemplate{frametitle}{}{%
\begin{textblock*}{100mm}(11.4cm,-1.3cm)
\includegraphics[height=1cm,width=1cm,keepaspectratio]{fig/ddclogotransparent.png}
\end{textblock*}}

\AtBeginSection[]{
  \begin{frame}
  \vfill
  \centering
  \begin{beamercolorbox}[sep=8pt,center,shadow=true,rounded=true]{title}
    \usebeamerfont{title}\insertsectionhead\par%
  \end{beamercolorbox}
  \vfill
  \end{frame}
}

\begin{document}
\title{Quantum Math}
\author{Brian Rashap}
\date{August 2025} 



\begin{frame}
\titlepage
\end{frame}

\section{Algebra}
\begin{frame}\frametitle{Algebra Overview}
\begin{itemize}
\item Functions
\item Transformations
\item Polynomials
\item Rational Functions
\item Exponentials and Logarithms
\end{itemize}
\end{frame}

\begin{frame}\frametitle{Cartesian Coordinates}
\begin{columns}
\begin{column}{4cm}
Cartesian coordinates is a system of describing the position of points in space using perpendicular axis lines that meet at a point called the origin. Any given point’s position can be described based on its distance from the origin along each axis.
\end{column}
\begin{column}{6cm}
\begin{center}
\includegraphics[height=7cm]{fig/plotting.jpg}
\end{center}
\end{column}
\end{columns}
\end{frame}

\begin{frame}\frametitle{Measuring Distance - Pythagorean Theorem}
\begin{columns}
\begin{column}{6cm}
Pythagorean Theorem:
\begin{center}
$a^2 + b^2 = c^2$
\end{center}

For example: \newline

\hspace*{10mm}$d^2 = 3^2 + 4^2$ \newline
\hspace*{10mm}$d^2 = 9 + 16 = 25$ \newline
\hspace*{10mm}$d = \sqrt{25} = 5$ \newline

More generally for two points $P(x_1,y_1)$ and $Q(x_2,y_2)$ \newline

\hspace*{10mm}$d^2 = (x_2-x_1)^2 + (y_2 - y_1)^2$ \newline
\hspace*{10mm}$d = \sqrt{(x_2-x_1)^2 + (y_2 - y_1)^2}$ \newline

Noting that $\mid a \mid = (a)^2$: \newline


\end{column}
\begin{column}{5cm}
\begin{center}
\includegraphics[width=4cm]{fig/pythag.png}
\includegraphics[width=4cm]{fig/pythag2.png}
\end{center}
\end{column}
\end{columns}
\end{frame}

\begin{frame}\frametitle{Midpoints and Intercepts}
\begin{columns}
\begin{column}{6cm}

Midpoint: \newline

\hspace*{10mm}$M = (\frac{x_1+x_2}{2},\frac{y_2+y_1}{2})$ \newline


Intercepts: \newline

Two key features of a graph are where the graph intersects the x and y axes, the x-intercept and y-intercept, respectively.


\end{column}
\begin{column}{5cm}
\begin{center}
\includegraphics[width=4cm]{fig/midpoint.png}
\includegraphics[width=4cm]{fig/intercept.png}
\end{center}
\end{column}
\end{columns}
\end{frame}


\begin{frame}\frametitle{The Circle}
\begin{columns}
\begin{column}{7.5cm}

A circle is a set of all points that are equidistant from a fixed point called the center $(h,k)$. The distance from any point on the cirecle to the center is called the radius ($r$)

$r = \sqrt{(x-h)^2 + (y-k)^2}$ \newline

Equation of a circle: \newline
Standard form: $(x-h)^2 + (y-k)^2 = r^2$ \newline

Expand binomials: $x^2-hx+h^2+y^2-ky+k^2 - r^2 = 0$ \newline

General form: $x^2+y^2-hx-ky+(h^2+k^2-r^2)=0$

\hspace*{10mm} or

$x^2 + y^2 + Ax + By + C = 0$

\end{column}
\begin{column}{3.5cm}
\begin{center}
\includegraphics[width=3cm]{fig/circle.png}
\end{center}
\end{column}
\end{columns}
\end{frame}



\begin{frame}\frametitle{Domain and Range}
\begin{columns}
\begin{column}{6cm}
\begin{center}
\includegraphics[width=6cm]{fig/domain-range.png}
\end{center}
\end{column}

\begin{column}{5cm}
A set of ordered pairs $(x,y)$ is called a relation in x and y. 
\begin{itemize}
\item The set of x-values in the ordered pairs is called the domain of the relations.
\item The set of y-values in the ordered pairs is called the range of the relations.
\end{itemize}
\end{column}
\end{columns}
\end{frame}


\begin{frame}\frametitle{Linear Equations with Two Variables}
A linear equation in variables $x$ and $y$ can be written in the standard form:
\begin{equation}
Ax + By = C
\end{equation}

However, it is more common to see it in slope-intercept form:
\begin{equation}
y = mx + b
\end{equation}
where, $m$ is the slope and $b$ is the y-intercept
\end{frame}

\begin{frame}\frametitle{Linear Conversion - Slope and Y-Intercept}
\begin{columns}
\begin{column}{6cm}
\begin{center}
\includegraphics[width=5cm]{fig/slope.jpg}

$y = mx + b$

where $m$ is slope and $b$ y-intercept. 
\end{center}
\end{column}

\begin{column}{5.2cm}
For example, given two points:
\begin{itemize}
\item $(x_1,y_1) = (1,2)$
\item $(x_2,y_2) = (5,4)$
\end{itemize}
Find slope
\begin{itemize}
\item $m = \frac{rise}{run} = \frac{4-2}{5-1} = \frac{1}{2}$
\end{itemize}
Find y-intercept
\begin{itemize}
\item $y_1 = m * x_1 + b$
\item $b = y_1 - (m * x_1$)
\item $b = 2 - (\frac{1}{2} * 1) = 1 \frac{1}{2}$
\end{itemize}
\end{column}
\end{columns}

\vspace{1cm}

Use this to find the conversion from Celsius to Fahrenheit.
\end{frame}

\begin{frame}\frametitle{Parallel and Perpendicular Lines}
\begin{center}
\includegraphics[width=5.5cm]{fig/parperp.jpg}
\end{center}
\end{frame}



\begin{frame}\frametitle{Linear Regression}
\begin{columns}
\begin{column}{6cm}
\begin{center}
\includegraphics[width=5cm]{fig/lsqr.png}
\end{center}
\end{column}

\begin{column}{5cm}
Consider a set of data: $(x_1,y_1),(x_2,y_2),(x_3,y_3),...,(x_n,y_n)$
\begin{itemize}
\item The least-squares regression line $\hat{y} = mx + b$, is a unique line that minimizes the sum of the squared vertical deviations from the the observed data points to the line.
\end{itemize}
\end{column}
\end{columns}

\vspace{1cm}

Use this to find the conversion from Celsius to Fahrenheit.
\end{frame}


\begin{frame}\frametitle{Recognizing Functions}
\begin{columns}
\begin{column}{4.5cm}
An algebraic function provides a "y-value" for every "x-value"
\begin{itemize}
\item Linear: $y = x + 2$
\item Quadratic: $y = x^2$
\item Periodic: $y = sin(x)$
\end{itemize}
\end{column}
\begin{column}{7cm}
\begin{center}
\includegraphics[width=7cm]{fig/basicfun.jpg}
\end{center}
\end{column}
\end{columns}
\end{frame}


\begin{frame}\frametitle{Vertical and Horizontal Shifts}
\begin{columns}
\begin{column}{5.0cm}
\begin{center}
\includegraphics[height=4.5cm]{fig/shiftV.png}
\end{center}
\end{column}
\begin{column}{6.0cm}
\begin{center}
\includegraphics[height=4.5cm]{fig/shiftH.png}
\end{center}
\end{column}
\end{columns}
\end{frame}

\begin{frame}\frametitle{Shrink and Expand}

\begin{center}
\includegraphics[height=6.0cm]{fig/stretchV.png}
\end{center}

\end{frame}


\begin{frame}\frametitle{X and Y Reflections}
\begin{columns}
\begin{column}{5.5cm}
\begin{center}
\includegraphics[height=4.5cm]{fig/reflectX.png}
\end{center}
\end{column}
\begin{column}{5.5cm}
\begin{center}
\includegraphics[height=4.5cm]{fig/reflectY.png}
\end{center}
\end{column}
\end{columns}
\end{frame}

\begin{frame}\frametitle{Summary - Transformations of Functions}
\begin{center}
\includegraphics[width=10cm]{fig/transform.jpg}
\end{center}
\end{frame}


\begin{frame}\frametitle{Piece-Wise Functions}
\begin{columns}
\begin{column}{6cm}
\begin{center}
\includegraphics[width=6cm]{fig/pwfunct1.png}
\end{center}
\end{column}
\begin{column}{5cm}
\begin{center}
\includegraphics[width=6cm]{fig/pwfunct2.jpg}
\end{center}
\end{column}
\end{columns}
\end{frame}

\begin{frame}\frametitle{Rate of Change}
Given points $(x_1,y_1)$ and $(x_2,y_2)$ as points on the graph of a function $f()$, if $f()$ is defined on the interval $[x_1,x_2]$, then the average rate of change is the slope of the secant\footnote{Secante comes from the latin secare meaning "to cut."} line containing $(x_1,f(x_1))$ and $(x_2,f(x_2))$. 

\begin{center}
\includegraphics[width=6cm]{fig/roc.png}
\end{center}
\end{frame}


\begin{frame}\frametitle{Difference Quotient}
Suppose we choose a value $x$ from the domain of $f()$ and a second value $x + h$, where $h \neq 0$, but very small.

\begin{center}
\includegraphics[width=5cm]{fig/roc2.png}
\end{center}

The difference quotient\footnote{The difference quotient is important to calculus, where the exact rate of change at a point is given by $ \lim_{h \rightarrow 0}(m) $}.
\begin{equation}
m = \frac{f(x+h) - f(x)}{(x+h) - x} = \frac{f(x+h) - f(x)}{h}
\end{equation}

\end{frame}

\begin{frame}\frametitle{Increaing, Decreaing, Constant}
\begin{center}
\includegraphics[height=8cm]{fig/idc.jpg}
\end{center}
\end{frame}

\begin{frame}\frametitle{Local Minima and Maxima}

\begin{itemize}
\item $f(a)$ is a relative maximum of $f$ if there exists an open interval\footnote{An open interval is an interval in which the endpoints are not included.} containing a such that $f(a) \geq f(x)$ for all $x$ in the interval.
\item $f(b)$ is a relative minimum of $f$ if there exists an open interval\footnote{An open interval is an interval in which the endpoints are not included.} containing a such that $f(b) \leq f(x)$ for all $x$ in the interval.
\end{itemize}
\begin{center}
\includegraphics[width=8cm]{fig/minmax.png}
\end{center}
\end{frame}


\begin{frame}\frametitle{Operations on Functions}

\end{frame}


\subsection{Polynomial Functions}

\begin{frame}\frametitle{Quadratic Function}

A quadratic function is often used as a model for the projectile motion. This is the motion followed by an object influenced by an initial force and by the force of gravity.

\begin{center}
\includegraphics[width=8cm]{fig/basketball.png}
\end{center}

\end{frame}



\begin{frame}\frametitle{Quadratic Function - Vertex Form}
\begin{columns}
\begin{column}{3.5cm}
\begin{center}
\includegraphics[height=3cm]{fig/quadup.png}

\vspace{0.25cm}

\includegraphics[height=3cm]{fig/quaddown.png}
\end{center}
\end{column}
\begin{column}{7.5cm}
Quadratic Function: $f(x) = ax_2 + bx + c (a \neq 0)$

By completing the square, it can be expressed in vertex form: $f(x) = a(x-h)^2 + k$
\begin{itemize}
\item The graft of f(x) is a parabola with vertex $(h,k)$
\item If $a>0$ the parabola opens upward and minimum value is $k$.
\item If $a>0$ the parabola opens downward and maximum value is $k$.
\item The axis of symmetry is $x=h$.
\end{itemize}
\end{column}
\end{columns}
\end{frame}


\begin{frame}\frametitle{Polynomial Functions}

\begin{center}
\includegraphics[width=10cm]{fig/poly1.jpg}
\end{center}

\end{frame}


\begin{frame}\frametitle{Even and Odd Exponents}
\begin{columns}
\begin{column}{5.5cm}

Even:
\begin{center}
\includegraphics[width=5cm]{fig/polyeven.png}
\end{center}



\end{column}

\begin{column}{5.5cm}
Odd:
\begin{center}
\includegraphics[width=5cm]{fig/polyodd.png}
\end{center}

\end{column}
\end{columns}

\end{frame}

\begin{frame}\frametitle{Polynomial End Behavior}

\begin{center}
\includegraphics[width=10cm]{fig/poly2.jpg}
\end{center}

\end{frame}

\begin{frame}\frametitle{Polynomial Finding Zeros}

\begin{center}

\end{center}

\end{frame}

\begin{frame}\frametitle{Rational Functions}

\begin{center}
\includegraphics[width=10cm]{fig/ratfun1.jpg}
\end{center}

\end{frame}



\begin{frame}\frametitle{Exponential Functions}
\begin{itemize}
\item Linear growth - a constant rate of change, that is,a constant number by which the output increased for each unit increase in input.
\item Exponential growth - increase based on a constant multiplicative rate of change over equal increments of time, that is, a percent increase of the original amount over time.
\end{itemize}

\begin{center}
\includegraphics[width=12cm]{fig/exp_growth.jpg}
\end{center}

\end{frame}


\begin{frame}\frametitle{Origami to the Moon}
\begin{center}
\includegraphics[width=12cm]{fig/40fold.jpg}
\end{center}
\end{frame}

\begin{frame}\frametitle{What about Negative Exponents}
\begin{columns}
\begin{column}{7cm}
The general form of an exponential function is $f(x) = ab^x$, where $a$ is any non-zero number and $b$ is an positive number not equal to 1.

\begin{itemize}
\item If $b > 1$ the function grows at a rate proportional to its size.
\item If $0 < b < 1$ the function decays at a rate proportional to its size.
\end{itemize}
\end{column}

\begin{column}{5cm}
For example, $f(x) = 2^x$:
\begin{center}
\includegraphics[width=4cm]{fig/exp2g.jpg}
\end{center}
\end{column}
\end{columns}

\begin{center}
\includegraphics[width=12cm]{fig/exp2t.jpg}
\end{center}
\end{frame}

\begin{frame}\frametitle{Scientific (SI) Prefixes}
\begin{center}
\includegraphics[width=12cm]{fig/si_prefixes.jpg}
\end{center}
\end{frame}

\begin{frame}\frametitle{$e$ - an interesting aside}
The letter e represents the irrational number:
\begin{equation}
e = (1 + \frac{1}{n})^n
\end{equation}
as $n$ increases without bound. \newline



The number $e$ is used as a base for many real-world exponential models. To work with base e, we use the approximation,  $e \approx 2.718282$. The constant was named by the Swiss mathematician Leonhard Euler (1707–1783) who first investigated and discovered many of its properties.
\end{frame}


\begin{frame}\frametitle{Graphing Exponentials}
\begin{columns}
\begin{column}{7cm}

Shifts:
\begin{center}
\includegraphics[width=6.5cm]{fig/exp2vert.jpg}
\end{center}

\begin{center}
\includegraphics[width=6.5cm]{fig/exp2hor.jpg}
\end{center}

\end{column}

\begin{column}{5cm}
Stretch:
\begin{center}
\includegraphics[width=2.8cm]{fig/exp2v.jpg}
\end{center}

Flip:
\begin{center}
\includegraphics[width =2.8cm]{fig/exp2f.jpg}
\end{center}
\end{column}
\end{columns}

\end{frame}

\begin{frame}\frametitle{Logarithmic Functions}

\begin{center}
\includegraphics[width=10cm]{fig/log1.jpg}
\end{center}

\end{frame}

\begin{frame}\frametitle{Natural Log}

\begin{center}
\includegraphics[width=10cm]{fig/logln.jpg}
\end{center}

\end{frame}

\begin{frame}\frametitle{Logarithmic Functions}

\begin{table}[H]
    \centering
    
  \begin{tabular}{|l|c|l|}
    \hline \hline
	Logarithmic form: $y =log_b x$ & & Exponential form: $b^y = x$ \\
   \hline  $log_2 16 = 4$ & $\Longleftrightarrow$ & $2^4 = 16$ \\
   \hline  $log_10 \frac{1}{100} = -2$ & $\Longleftrightarrow$ & $10^{-2} = \frac{1}{100}$ \\
   \hline  $log_7 1 = 0$ & $\Longleftrightarrow$ & $7^0 = 1$ \\
\hline \hline
  \end{tabular}
\end{table}

\end{frame}

\begin{frame}\frametitle{Graphing Exponential and Logarithmic Functions }

\begin{center}
\includegraphics[width=10cm]{fig/loggraph.jpg}
\end{center}

\end{frame}


\begin{frame}\frametitle{Logarithmic Transforms}
\begin{columns}
\begin{column}{6cm}


\begin{center}
\includegraphics[width=5cm]{fig/logtran1.png}
\end{center}


\end{column}

\begin{column}{6cm}

\begin{center}
\includegraphics[width=5cm]{fig/logtran2.png}
\end{center}

\end{column}
\end{columns}

\end{frame}

\begin{frame}\frametitle{Logarithm Rules}
\begin{itemize}
\item Product Property

\begin{equation}
log_b(xy) = log_b(x) + log_b(y)
\end{equation}

\item Quotient Property

\begin{equation}
log_b(\frac{x}{y}) = log_b(x) - log_b(y)
\end{equation}

\item Power Property

\begin{equation}
log_b x^p = p \cdot log_b x
\end{equation}

\end{itemize}
\end{frame}


\begin{frame}\frametitle{Properties of Logarithms}
\begin{center}
\includegraphics[width=10cm]{fig/logprop.jpg}
\end{center}
\end{frame}

\section{Trigonometry}



\begin{frame}\frametitle{Algebraic Functions}
\begin{columns}
\begin{column}{4.5cm}
An algebraic function provides a "y-value" for every "x-value"
\begin{itemize}
\item Linear: $y = x + 2$
\item Quadratic: $y = x^2$
\item Periodic: $y = sin(x)$
\end{itemize}
\end{column}
\begin{column}{7cm}
\begin{center}
\includegraphics[width=7cm]{fig/basicfun.jpg}
\end{center}
\end{column}
\end{columns}
\end{frame}

\begin{frame}\frametitle{More Desmos Fun}

\begin{center}
\includegraphics[width=12cm]{fig/desmosfun.jpg}
\end{center}
\end{frame}


\begin{frame}\frametitle{Pi ($\pi)$}

\begin{center}
\includegraphics[width=8cm]{fig/pi.png}
\end{center}

\end{frame}


\begin{frame}\frametitle{Unit Circle and Trigonometric Functions}
The Unit Circle is a circle with a radius of 1.
\begin{columns}
\begin{column}{6cm}
\begin{center}
\includegraphics[scale=0.75]{fig/unitcircle.png}
\end{center}
\end{column}
\begin{column}{6cm}
\begin{center}
\includegraphics[scale=0.75]{fig/unitcircle_cst.png}
\end{center}
\end{column}
\end{columns}
The Unit Circle can be used to map out the trigonometric values of sine, cosine, and tangent.
\end{frame}


\begin{frame}\frametitle{Unit Circle and the Value of $sin(\theta)$}
\begin{columns}
\begin{column}{6cm}
\begin{center}
\includegraphics[scale=0.25]{fig/unitcircle_sin.png}
\end{center}
\end{column}
\begin{column}{6cm}
\begin{center}
\includegraphics[scale=0.75]{fig/unitcircle_rad.png}
\end{center}
\end{column}
\end{columns}

\vspace{0.25cm}

\begin{itemize}
\item $sin(\theta)$ is the y-value of the point on the Unit Circle at angle $\theta$. 
\item In our trig functions, $\theta$ is measured in radians (rad), not degrees.
\item 360 degrees = $2 \pi$ radians.
\end{itemize}
\end{frame}



\begin{frame}\frametitle{Sine Waves}
\begin{center}
\includegraphics[scale=0.25]{fig/sin.jpg}
\end{center}

\begin{center}
$y = A * sin(2*\pi*\nu*t) + B$
\end{center}

where A = amplitude, B = offset, $\nu$ = frequency = $\frac{1}{period}$, 

and t = time in seconds.
\end{frame}


\begin{frame}\frametitle{Using Desmos (desmos.com/calculator)}
\begin{center}
\includegraphics[scale=0.35]{fig/desmos1.jpg}
\end{center}
\begin{center}

\vspace{0.5cm}

\includegraphics[scale=0.35]{fig/desmos2.jpg}
\end{center}
\end{frame}

\begin{frame}\frametitle{Phase Shift}
The sine wave can be shifted relative to each other by adding in a phase shift ($\phi$), which will shift the wave to the left or right.

Blue lags Red: 
\begin{center}
\includegraphics[height=2.6cm]{fig/sin_lag.jpg}
\end{center}

Green leads Red: 
\begin{center}
\includegraphics[height=2.6cm]{fig/sin_lead.jpg}
\end{center}

\end{frame}




\begin{frame}\frametitle{SOH CAH TOA}

\begin{itemize}
\item sin = opposite over hypotenuse
\item cos = adjacent over hypotenuse
\item tan = opposite over adjacent
\end{itemize}

\vspace{0.25cm}

\begin{center}
\includegraphics[scale=0.35]{fig/sohcahtoa.jpg}
\end{center}
\end{frame}

\section{Vectors}

\begin{frame}\frametitle{Scalars and Vectors}
Scalars are quantities that are fully described by a magnitude (or numerical value) alone.

Vectors are quantities that are fully described by both a magnitude and a direction.

\vspace{0.25cm}
\begin{center}
\includegraphics[width=8cm]{fig/scalarVector.jpg}
\end{center}
\end{frame}


\begin{frame}\frametitle{Vector Components}
\begin{columns}
\begin{column}{6cm}
Finding the components of a vector involves forming a right triangle and using trigonometry's SOH-CAH-TOA
\begin{itemize}
\item $A_x = A cos(\theta)$
\item $A_y = A sin(\theta)$
\end{itemize}
\end{column}
\begin{column}{5cm}

\begin{center}
\includegraphics[width=4.8cm]{fig/vec5.png}
\end{center}
\end{column}
\end{columns}
\end{frame}


\begin{frame}\frametitle{Graphical Vector Addition}
\begin{columns}
\begin{column}{6cm}
Adding two vectors A and B graphically can be visualized like two successive walks, with the vector sum being the vector distance from the beginning to the end point. Representing the vectors by arrows drawn to scale, the beginning of vector B is placed at the end of vector A. The vector sum R can be drawn as the vector from the beginning to the end point.
\end{column}
\begin{column}{5cm}

\begin{center}
\includegraphics[width=4.8cm]{fig/vec1.png}
\end{center}
\end{column}
\end{columns}
\end{frame}


\begin{frame}\frametitle{Vector Components}
\begin{columns}
\begin{column}{6cm}
Finding the components of a vector involves forming a right triangle and using trigonometry's SOH-CAH-TOA
\begin{itemize}
\item Add the X components
\begin{itemize}
\item $A_x = 12 cos(20^\circ) = 11.3$
\item $B_x = 25 cos(60^\circ) = 12.5$
\item $R_x = A_x + B_x = 23.8$
\end{itemize}
\item Add the Y components
\begin{itemize}
\item $A_y = 12 sin(20^\circ) = 4.1$
\item $B_y = 25 sin(60^\circ) = 21.7$
\item $R_y = A_y + B_y = 25.8$
\end{itemize}
\end{itemize}
\end{column}
\begin{column}{5cm}

\begin{center}
\includegraphics[width=4.8cm]{fig/vec2a.png}
\end{center}
\end{column}
\end{columns}
\end{frame}

\begin{frame}\frametitle{Polar Form}
\begin{columns}
\begin{column}{6cm}
After finding the components, the result can be placed in Polar Form:
\begin{itemize}
\item $R_x = 23.8$
\item $R_y = 25.8$
\item $R = \sqrt{R_x^2 + R_y^2} = 35.05$
\item $\theta_R = tan^{-1} (\frac{R-y}{R_x}) = 47.3^{\circ}$
\end{itemize}
\end{column}
\begin{column}{5cm}

\begin{center}
\includegraphics[width=4.8cm]{fig/vec4a.png}
\end{center}
\end{column}
\end{columns}
\end{frame}

\begin{frame}\frametitle{Unit Vectors in 3 Dimensions}
\begin{columns}
\begin{column}{6cm}
Vectors of unit length (i.e., length equals 1) can be used to specify the direction of vector quantities in various coordinate systems.

In Cartesian coordinates, it is typical to use i, j, and k to represent the unit vectors in the x, y, and z directions, respectively: 

\begin{equation}
\vec{r} = x \vec{i} + y \vec{j} + z \vec{k}
\end{equation}
\end{column}
\begin{column}{5cm}

\begin{center}
\includegraphics[width=4.8cm]{fig/uvec.jpg}
\end{center}
\end{column}
\end{columns}
\end{frame}

\begin{frame}\frametitle{Dot (Scalar) Product}
\begin{columns}
\begin{column}{6cm}
The dot (or inner or scalar) product of two vectors can be constructed by taking the component of the first vector in the direction  of the second vector and by multiplying it by the second vector's magnitude.

\begin{equation}
\vec{A} \cdot \vec{B} = AB cos(\theta)
\end{equation}

or

\begin{equation}
\vec{A} \cdot \vec{B} = A_x B_x + A_y B_y + A_z B_y
\end{equation}
\end{column}
\begin{column}{5cm}

\begin{center}
\includegraphics[width=4.8cm]{fig/vecdot.jpg}
\end{center}
\end{column}
\end{columns}
\end{frame}

\begin{frame}\frametitle{Cross (Vector) Product}
\begin{columns}
\begin{column}{6cm}
The magnitude of the cross (or outer or vector) product of vectors can be constructed by taking the product of the magnitude of the vectors times the sine of the angle between them.
\begin{equation}
\vec{A} \times \vec{B}_{magnitude} = AB sin(\theta)
\end{equation}

and the direction is given by the right hand rule.

\end{column}
\begin{column}{5cm}

\begin{center}
\includegraphics[width=4.8cm]{fig/veccross.jpg}

\includegraphics[width=2.8cm]{fig/righthandrule.png}
\end{center}
\end{column}
\end{columns}
In terms of unit vectors:

\begin{equation}
\vec{A} \times \vec{B} = \vec{i}(A_yB_z-A_zB_y) + \vec{j}(A_zB_x-A_xB_z) + \vec{k}(A_xB_y-A_yB_x)
\end{equation}
\end{frame}


\begin{frame}\frametitle{Bonus - Cross Product - Determinant Form}

THe cross product can be compactly stated in teh form of a determinant:

\[
\vec{A} \times \vec{B} =
\begin{vmatrix}
\vec{i} & \vec{j} & \vec{k} \\
A_x & A_y & A_z \\
B_x & B_y & B_z
\end{vmatrix}
\]

Which can be expanded to

\[
\vec{A} \times \vec{B} = \vec{i}(A_yB_z-A_zB_y) + \vec{j}(A_zB_x-A_xB_z) + \vec{k}(A_xB_y-A_yB_x)
\]
\end{frame}




\begin{frame}\frametitle{Pythagorean Theorem in 3 Dimensions}
\begin{columns}
\begin{column}{5cm}
\begin{center}
\includegraphics[width=5cm]{fig/pathagorean.png}

\vspace{0.25cm}

\includegraphics[width=4cm]{fig/pathag3D.jpg}

\end{center}
\end{column}
\begin{column}{5cm}
To add orthogonal (at right angles to each other) vectors in 3 Dimensions:
\begin{itemize}
\item $C = \sqrt{A_x^2 + A_y^2}$
\item $A_{total} = \sqrt{C^2 + A_z^2}$
\item $A_{total} = \sqrt{A_x^2 + A_y^2 + A_z^2}$
\end{itemize}
\end{column}
\end{columns}
\end{frame}





\section{Probability and Statistics}

\begin{frame}\frametitle{Probability and Statistics}
\begin{columns}
\begin{column}{5cm}
\begin{center}
\includegraphics[width=5cm]{fig/cartoonguidestats.jpg}
\end{center}
\end{column}
\begin{column}{5cm}
\begin{itemize}
\item Data Analysis: The gathering, display, and summary of data
\item Probability: The laws of chance (inside and outside of a casino)
\item Statistical Inference: the science of drawing statistical conclusions from specific data using the laws of probability
\end{itemize}
\end{column}
\end{columns}
\end{frame}

\begin{frame}\frametitle{A Tale as Old as Time}
\begin{center}
\includegraphics[width=6cm]{fig/vegas.jpg}
\end{center}


Gambling is as old as mankind, so it seems that probability should be almost as old. But, the realization that one could predict an outcome to a certain degree of accuracy was unconceivable until the $16^{th}$ century. In order to make a profit, underwriters were in need of dependable guidelines by which a profit could be expected, while the gambler was interested in predicting the possibility of gain.
\end{frame}



\begin{frame}\frametitle{Rich Guys Gambling}
\begin{columns}
\begin{column}{4cm}
\begin{center}
\includegraphics[width=3cm]{fig/cardano.jpg}
\end{center}
\end{column}
\begin{column}{7.5cm}
Known as the “Father of Probability”, Gerolamo Cardano was an Italian mathematician, physician, and gambler who first talked about probability. Cardano’s fascination with games of chance led him to write the first book dedicated to probability, “Liber de Ludo Aleae” (Book on Games of Chance), published in 1564. Cardano introduced concepts like odds and probabilities in this work, providing a framework for analyzing the likelihood of different outcomes in dice rolls and other games.
\end{column}
\end{columns}
\end{frame}

\begin{frame}\frametitle{Probability}


\begin{columns}
\begin{column}{5cm}
The probability of an event expresses the likelihood of the event outcome

\[P(event) = \frac{\text{\# of favorable outcomes}}{\text{\# of all possibly outcomes}}\]
\end{column}
\begin{column}{5cm}
\begin{center}
\includegraphics[width=3cm]{fig/coinflip.jpg}
\end{center}
\end{column}
\end{columns}
\end{frame}

\begin{frame}\frametitle{Frequency Tables and Histograms}

\begin{center}

Roll two dice and add them together, repeat. \newline

\includegraphics[width=10cm]{fig/stat1.jpg}
\end{center}

\end{frame}

\begin{frame}\frametitle{Assignment: Tenzi}
\begin{columns}
\begin{column}{4cm}
\begin{center}
\includegraphics[width=4cm]{fig/dice.jpg}
\end{center}
\end{column}
\begin{column}{7cm}

Role two dice 12 times, after each role:
\begin{itemize}
\item Record the sum
\item Record the running average (total up to this point divided by number of roles)
\end{itemize}
After the twelfth role:
\begin{itemize}
\item Create a histogram of the sums
\item Create a line chart of the running averages
\end{itemize}
\end{column}
\end{columns}
\end{frame}


\begin{frame}\frametitle{Summary Statistics}


\begin{columns}
\begin{column}{7cm}
\begin{itemize}
\item Mean (or average): add the totals and divide by number of samples
\[ \bar{x} = \frac{x_1 + x_2 + x_3 + ... + x_n}{n} \]
or
\[ \bar{x} = \frac{\sum_{i=1}^{n} (x_i)}{n}\]
\item Median: middle sample in the ordered distribution
\begin{itemize}
\item Odd: median is the value of middle sample
\item Even: median is the average of the two middle samples
\end{itemize}
\item Mode: the value the appears most often
\end{itemize}
\end{column}
\begin{column}{4cm}
\begin{center}
\includegraphics[width=4cm]{fig/stat2.jpg}


average = $\bar{x} = 7$

median $= 6$

mode $ = 6$
\end{center}
\end{column}
\end{columns}
\end{frame}


\begin{frame}\frametitle{Summary Statistics: Variation}
\begin{center}
\includegraphics[width=11cm]{fig/stat3.jpg}
\end{center}
\begin{itemize}
\item Interquartile Range (IRQ): Spread of the middle $50\%$ of the data
\begin{itemize}
\item Find the median
\item Find the first quartile (Q1): median of the lower half.
\item Find the third quartile (Q3): median of the upper half.
\item IRQ = Q3 - Q1 = 6
\end{itemize}


\item Standard Deviations ($\sigma$):
\[ \sigma = \sqrt{ \frac{ \sum_{i=0}^n (x_i - \bar{x})^2}{n}} = 3.06\]

\end{itemize}

\end{frame}


\begin{frame}\frametitle{Gaussian Distribution}

\begin{center}
\includegraphics[width=6cm]{fig/stat4.png}
\end{center}

Probability Distribution Function $f(x)$:

\[ f(x) = \frac{1}{\sqrt{2 \pi \sigma^2}} e^{\frac{-(x-\bar{x})^2}{2 \sigma^2}} \]

\end{frame}

\begin{frame}\frametitle{Gaussian Distribution: $3 \sigma$}
\begin{columns}
\begin{column}{6cm}
\begin{center}
\includegraphics[width=6cm]{fig/stat5.png}
\end{center}
\end{column}
\begin{column}{4cm}

\begin{center}
Galton Baord
\includegraphics[width=4cm]{fig/stat7.png}
\end{center}
\end{column}
\end{columns}
\end{frame}

\begin{frame}\frametitle{Assignment: Measuring Bean Profile}
\begin{columns}
\begin{column}{4cm}
\begin{center}
\includegraphics[width=3cm]{fig/profile1.jpg}
\includegraphics[width=3cm]{fig/profile2.jpg}
\end{center}
\end{column}
\begin{column}{7cm}
The laser beam has a Gaussian profile.
\begin{enumerate}
\item Direct a laser beam at a power meter, setting the correct wavelength
\item Moving the knife edge across the beam path, record the drop in power vs distance.
\item Setup a telescope to expand the beam size by 2-5 times.
\item Repeat the knife edge process.
\item Plot the results of both sets of measurements on paper and in excel.
\end{enumerate}
\end{column}
\end{columns}
\end{frame}


\section{Electrical Components and Circuits}

\begin{frame}
\frametitle{Introduction to Electrical Circuits}
\begin{figure}
\includegraphics[scale=0.15]{fig/circuits2.png} 
\end{figure}
\end{frame}

\begin{frame}
\frametitle{Energy}
\begin{figure}
\includegraphics[scale=0.15]{fig/EnergyAlien.jpg} 
\end{figure}
\begin{itemize} 
\item Kinetic Energy - energy of motion
\item Potential Energy - energy stored in an object
\end{itemize}
\end{frame}

\begin{frame}\frametitle{Electrical Circuit Terms}
\begin{figure}
\includegraphics[scale=0.50]{fig/terms.png} 
\end{figure}
\end{frame}

\begin{frame}
\frametitle{Measuring Voltage, Current, and Resistance}
\begin{figure}
\includegraphics[scale=0.65]{fig/multi1.jpg} 
\includegraphics[scale=0.38]{fig/multi2.jpg} 
\end{figure}
\end{frame}

\begin{frame}\frametitle{Ohm's Law}
Georg  Ohm (16 March 1789 – 6 July 1854) was a German physicist and mathematician. As a school teacher, Ohm began his research with the new electrochemical cell, invented by Italian scientist Alessandro Volta. Ohm found that there is a direct proportionality between the potential difference (voltage) applied across a conductor and the resultant electric current. This relationship is known as Ohm's law:
\vspace{0.5cm}
\begin{columns}
\begin{column}{5.5cm}
\includegraphics[scale=0.33]{fig/ol2.jpg}
\end{column}
\begin{column}{5.5cm}
\includegraphics[scale=.1]{fig/Volt_Amp_Ohm.png}
\end{column}
\end{columns}
\end{frame}


\begin{frame}\frametitle{A word about circuit notation}
\begin{columns}
\begin{column}{2.5cm}
\begin{center}
\includegraphics[width=1.5cm]{fig/notation.png}
\end{center}
\end{column}
\begin{column}{8.5cm}
Subscripts will be used to denote quanties (voltage, currect, etc) for different elements:
\begin{itemize}
\item i1 or $i_1$ is the current through Resistor 1 ($R_1$) 
\item i2 or $i_2$ is the current through Resistor 2 ($R_2$) 
\item v2 or $v_2$ is the votage across Resistor 2 ($R_2$) 
\item $I$ is the current delivered by the power supply
\item $V_{cc}$ (common collector \footnote{Common Collector is a term for certain parts of transistor circuits. We will learn about Transistors in Lesson 11} voltage) is the notation will will use for $3.3V$ from the Particle
\end{itemize}
\end{column}
\end{columns}
\end{frame}


\begin{frame}
\frametitle{Kirchhoff's First Law}
Gustav Robert Kirchhoff (12 March 1824 – 17 October 1887) was a German physicist who contributed to the fundamental understanding of electrical circuits. His first law:
\begin{columns}
\begin{column}{5cm}
In an electrical circuit, the sum of currents flowing into that node is equal to the sum of currents flowing out of that node
\end{column}
\begin{column}{4cm}
\begin{overprint}
\includegraphics[scale=0.25]{fig/KFL2.png}
\end{overprint}
\end{column}
\end{columns}
\end{frame}

\begin{frame}
\frametitle{Kirchhoff's Second Law}
\begin{columns}
\begin{column}{4cm}
The directed sum of the potential differences (voltages) around any closed loop is zero.
\end{column}
\begin{column}{6cm}
\begin{overprint}
\includegraphics[scale=0.25]{fig/KSL2.png}
\end{overprint}
\end{column}
\end{columns}
\end{frame}

\begin{frame}
\frametitle{Kirchhoff's Second Law}
\begin{figure}
\includegraphics[scale=0.40]{fig/klooplaw.jpg} 
\end{figure}
\end{frame}


\begin{frame}
\frametitle{Resistors in Series and Parallel}
\begin{figure}
\includegraphics[scale=0.40]{fig/SP.png} 
\end{figure}

\vspace{0.25cm}

How many nodes? How many loops?

\end{frame}

\begin{frame}
\frametitle{Resistors in Series and Parallel}
\begin{columns}
\begin{column}{5cm}
\begin{center}
\includegraphics[scale=0.25]{fig/series.png}

\vspace{0.5cm}

$R_{eq} = R_1+R_2+R_3$
\end{center}
\end{column}
\begin{column}{5cm}
\begin{center}
\includegraphics[scale=0.25]{fig/parallel.png}

\vspace{0.5cm}

$\frac{1}{R_{eq}} = \frac{1}{R_1}+\frac{1}{R_2}$
\end{center}
\end{column}
\end{columns}
\end{frame}

\begin{frame}\frametitle{Resistors in Series}
\begin{columns}
\begin{column}{5cm}
\begin{center}
\includegraphics[scale=0.25]{fig/series.png}

\vspace{0.5cm}

Node Law: $I = I_1 = I_2 = I_3$

\vspace{0.2cm}

Loop Law: $V_{cc} - (V_1 + V_2 + V_3) = 0$

\end{center}
\end{column}
\begin{column}{5cm}

Rearranging the Loop Law:
\begin{equation}
V_{cc} = V_1 + V_2 + V_3
\end{equation}

Using Ohm's Law:
\begin{equation}
V_{cc} = I R_1 + I R_2 + I R_3
\end{equation}

Using the Distributive Property:
\begin{equation}
V_{cc} = I (R_1 + R_2 + R_3)
\end{equation}

Gives the Equivalent Resistance:
\begin{equation}
\boxed{R_{eq} = R_1 + R_2 + R_3}
\end{equation}

\end{column}
\end{columns}
\end{frame}

\begin{frame}\frametitle{Resistors in Parallel}
\begin{columns}
\begin{column}{5cm}
\begin{center}
\includegraphics[scale=0.25]{fig/parallel.png}

\vspace{0.5cm}

Node Law: $I = I_1 + I_2$

\vspace{0.2cm}


Loop Law: $V_{cc} = V_1 = V_2$

\end{center}
\end{column}
\begin{column}{5cm}

\begin{equation}
I = I_1 + I_2
\end{equation}

\begin{equation}
I = \frac{V_1}{R_1} + \frac{V_2}{R_2}
\end{equation}

\begin{equation}
I = \frac{V_{cc}}{R_1} + \frac{V_{cc}}{R_2}
\end{equation}

\begin{equation}
\frac{V_{cc}}{R_{eq}} = V_{cc} (\frac{1}{R_1} + \frac{1}{R_2})
\end{equation}

\begin{equation}
\boxed{\frac{1}{R_{eq}} = (\frac{1}{R_1} + \frac{1}{R_2})}
\end{equation}
\end{column}
\end{columns}
\end{frame}


\begin{frame}\frametitle{Switches}
\begin{figure}[h]
	\includegraphics[scale=0.40]{fig/switchstate2.png}
\end{figure}
\end{frame}

\begin{frame}\frametitle{Poles and Throws}
\begin{figure}[h]
	\includegraphics[scale=0.25]{fig/st2.jpg}
\end{figure}
\begin{itemize}
\item Poles indicates the number of circuits that one switch can control for one operation of the switch. 
\item Throws indicates the number of contact points.
\end{itemize}
\end{frame}


\begin{frame}\frametitle{A Button - SPST}
\begin{figure}[h]
	\includegraphics[scale=0.20]{fig/SPST2.png}
\end{figure}
\end{frame}


\begin{frame}\frametitle{Inductors}
\begin{columns}
\begin{column}{4cm}
\begin{center}
\includegraphics[scale=0.7]{fig/bsav3.png}

\vspace{0.25cm}

\includegraphics[scale=0.7]{fig/sol3.png}

\vspace{0.25cm}

\includegraphics[scale=1.00]{fig/ind.png}

\end{center}
\end{column}
\begin{column}{6.5cm}
\begin{itemize}
\item Current flowing in a wire produces a magnetic field (B) around the wire (from Ampere's Law).
\item Wire wrapped into a coil produces a magnetic field that resembles a bar magnet through the center of the coil.
\item Also, in a coil, this magnetic field produces an effect known as Inductance (L) that opposes changes in electric current. 
\end{itemize}
\end{column}
\end{columns}
\end{frame}


\begin{frame}\frametitle{Relays}
\begin{columns}
\begin{column}{5cm}
\begin{center}
\includegraphics[scale=0.25]{fig/relayschem.jpg}

\vspace{0.25cm}

\includegraphics[scale=0.25]{fig/emitterfollower.png}
\end{center}
\end{column}
\begin{column}{6cm}
\begin{itemize}
\item When a device (e.g. a pump) requires higher voltage ($>5V$) or higher current, then a relay can be used as a switch for the device
\item The relay is activated by a digital pin from the microcontroller. 
\begin{itemize}
\item As the relay can require as much as 100mA from the digital pin, to provide sufficient current, use a current amplifying emitter follower to draw current directly from the USB connection ($V_{BUS}$).
\end{itemize}
\end{itemize}
\end{column}
\end{columns}
\end{frame}

\begin{frame}\frametitle{Optocoupled Relay}
\begin{center}
\includegraphics[scale=0.40]{fig/optorelay.png}
\end{center}
\begin{columns}
\begin{column}{5cm}
\includegraphics[scale=0.15]{fig/relay3va.jpg}
\end{column}
\begin{column}{5cm}
\begin{itemize}
\item Optocoupler isolates the relay load (which could be up to 240V) from the microcontroller electronics.
\end{itemize}
\end{column}
\end{columns}
\end{frame}

\begin{frame}\frametitle{Electrical Safety - Three Wires}
\begin{center}
\includegraphics[scale=0.40]{fig/lng.jpg}
\end{center}
\end{frame}

\begin{frame}\frametitle{Electrical Safety - Electric Shock}
\begin{center}
\includegraphics[scale=0.40]{fig/inductedemf.jpg}
\end{center}
\end{frame}

\begin{frame}\frametitle{Electrical Safety - GFCI}
\begin{center}
\includegraphics[scale=0.80]{fig/gfci.png}
\end{center}
\end{frame}

\begin{frame}\frametitle{Electrical Safety - 240V}
\begin{center}
\includegraphics[scale=1.8]{fig/hse.png}
\end{center}
\end{frame}



\end{document}
