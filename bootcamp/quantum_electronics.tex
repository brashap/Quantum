\documentclass{beamer}
\setbeamertemplate{navigation symbols}{}
\usepackage{comment}

\setbeamercolor{frametitle}{fg=black,bg=white}
\setbeamercolor{title}{fg=black,bg=yellow!85!orange}
\usetheme{AnnArbor}

\usepackage{textpos} % package for the positioning
\usepackage{listings}
\usepackage{xcolor}
\usepackage[most]{tcolorbox}
\usepackage{mathtools}
\usepackage{graphicx}
\usepackage{graphbox}
\usepackage{caption}
\DeclareCaptionType{code}[Code Listing][List of Code Listings] 

\definecolor{codegreen}{rgb}{0,0.6,0}
\definecolor{codegray}{rgb}{0.5,0.5,0.5}
\definecolor{codepurple}{rgb}{0.58,0,0.82}
\definecolor{backcolour}{rgb}{0.95,0.95,0.92} 
\lstdefinestyle{mystyle}{
    backgroundcolor=\color{backcolour},   
    commentstyle=\color{codegreen},
    keywordstyle=\color{magenta},
    numberstyle=\tiny\color{codegray},
    stringstyle=\color{codepurple},
    basicstyle=\ttfamily\footnotesize,
    breakatwhitespace=false,         
    breaklines=true,                 
    captionpos=b,                    
    keepspaces=true,                 
    numbers=left,                    
    numbersep=5pt,                  
    showspaces=false,                
    showstringspaces=false,
    showtabs=false,                  
    tabsize=2
}

\lstset{style=mystyle}

\lstdefinelanguage
   [x64]{Assembler}     % add a "x64" dialect of Assembler
   [x86masm]{Assembler} % based on the "x86masm" dialect
   % with these extra keywords:
   {morekeywords={CDQE,CQO,CMPSQ,CMPXCHG16B,JRCXZ,LODSQ,MOVSXD, %
                  POPFQ,PUSHFQ,SCASQ,STOSQ,IRETQ,RDTSCP,SWAPGS, %
                  rax,rdx,rcx,rbx,rsi,rdi,rsp,rbp, %
                  r8,r8d,r8w,r8b,r9,r9d,r9w,r9b, %
                  r10,r10d,r10w,r10b,r11,r11d,r11w,r11b, %
                  r12,r12d,r12w,r12b,r13,r13d,r13w,r13b, %
                  r14,r14d,r14w,r14b,r15,r15d,r15w,r15b}} %


\beamersetuncovermixins{\opaqueness<1>{25}}{\opaqueness<2->{15}}

%Copyright
\addtobeamertemplate{frametitle}{}{%
\begin{textblock*}{50mm}(0cm,-1.25cm)
\color{yellow!85!orange}
\tiny{Copyright \copyright 2024 CNM.}
\end{textblock*}}

% position the logo
\addtobeamertemplate{frametitle}{}{%
\begin{textblock*}{100mm}(11.4cm,-1.3cm)
\includegraphics[height=1cm,width=1cm,keepaspectratio]{fig/ddclogotransparent.png}
\end{textblock*}}

\AtBeginSection[]{
  \begin{frame}
  \vfill
  \centering
  \begin{beamercolorbox}[sep=8pt,center,shadow=true,rounded=true]{title}
    \usebeamerfont{title}\insertsectionhead\par%
  \end{beamercolorbox}
  \vfill
  \end{frame}
}

\begin{document}
\title{Quantum Electronics and Circuits}
\author{Brian Rashap}
\date{November 2025} 



\begin{frame}
\titlepage
\end{frame}


\section{Electrical Components and Circuits}

\begin{frame}
\frametitle{Resistive Circuits}
\begin{figure}
\includegraphics[scale=0.15]{fig/circuits2.png} 
\end{figure}
\end{frame}

\begin{frame}
\frametitle{Energy}
\begin{figure}
\includegraphics[scale=0.15]{fig/EnergyAlien.jpg} 
\end{figure}
\begin{itemize} 
\item Kinetic Energy - energy of motion
\item Potential Energy - energy stored in an object
\end{itemize}
\end{frame}

\begin{frame}\frametitle{Electrical Circuit Terms}
\begin{figure}
\includegraphics[scale=0.50]{fig/terms.png} 
\end{figure}
\end{frame}

\begin{frame}
\frametitle{Measuring Voltage, Current, and Resistance}
\begin{figure}
\includegraphics[scale=0.65]{fig/multi1.jpg} 
\includegraphics[scale=0.38]{fig/multi2.jpg} 
\end{figure}
\end{frame}

\begin{frame}\frametitle{Ohm's Law}
Georg  Ohm (16 March 1789 – 6 July 1854) was a German physicist and mathematician. As a school teacher, Ohm began his research with the new electrochemical cell, invented by Italian scientist Alessandro Volta. Ohm found that there is a direct proportionality between the potential difference (voltage) applied across a conductor and the resultant electric current. This relationship is known as Ohm's law:
\vspace{0.5cm}
\begin{columns}
\begin{column}{5.5cm}
\includegraphics[scale=0.33]{fig/ol2.jpg}
\end{column}
\begin{column}{5.5cm}
\includegraphics[scale=.1]{fig/Volt_Amp_Ohm.png}
\end{column}
\end{columns}
\end{frame}


\begin{frame}\frametitle{A word about circuit notation}
\begin{columns}
\begin{column}{2.5cm}
\begin{center}
\includegraphics[width=1.5cm]{fig/notation.png}
\end{center}
\end{column}
\begin{column}{8.5cm}
Subscripts will be used to denote quanties (voltage, currect, etc) for different elements:
\begin{itemize}
\item i1 or $i_1$ is the current through Resistor 1 ($R_1$) 
\item i2 or $i_2$ is the current through Resistor 2 ($R_2$) 
\item v2 or $v_2$ is the votage across Resistor 2 ($R_2$) 
\item $I$ is the current delivered by the power supply
\item $V_{cc}$ (common collector \footnote{Common Collector is a term for certain parts of transistor circuits. We will learn about Transistors in Lesson 11} voltage) is the notation will will use for $3.3V$ from the Particle
\end{itemize}
\end{column}
\end{columns}
\end{frame}


\begin{frame}
\frametitle{Kirchhoff's First Law}
Gustav Robert Kirchhoff (12 March 1824 – 17 October 1887) was a German physicist who contributed to the fundamental understanding of electrical circuits. His first law:
\begin{columns}
\begin{column}{5cm}
In an electrical circuit, the sum of currents flowing into that node is equal to the sum of currents flowing out of that node
\end{column}
\begin{column}{4cm}
\begin{overprint}
\includegraphics[scale=0.25]{fig/KFL2.png}
\end{overprint}
\end{column}
\end{columns}
\end{frame}

\begin{frame}
\frametitle{Kirchhoff's Second Law}
\begin{columns}
\begin{column}{4cm}
The directed sum of the potential differences (voltages) around any closed loop is zero.
\end{column}
\begin{column}{6cm}
\begin{overprint}
\includegraphics[scale=0.25]{fig/KSL2.png}
\end{overprint}
\end{column}
\end{columns}
\end{frame}

\begin{frame}
\frametitle{Kirchhoff's Second Law}
\begin{figure}
\includegraphics[scale=0.40]{fig/klooplaw.jpg} 
\end{figure}
\end{frame}


\begin{frame}
\frametitle{Resistors in Series and Parallel}
\begin{figure}
\includegraphics[scale=0.40]{fig/SP.png} 
\end{figure}

\vspace{0.25cm}

How many nodes? How many loops?

\end{frame}

\begin{frame}
\frametitle{Resistors in Series and Parallel}
\begin{columns}
\begin{column}{5cm}
\begin{center}
\includegraphics[scale=0.25]{fig/series.png}

\vspace{0.5cm}

$R_{eq} = R_1+R_2+R_3$
\end{center}
\end{column}
\begin{column}{5cm}
\begin{center}
\includegraphics[scale=0.25]{fig/parallel.png}

\vspace{0.5cm}

$\frac{1}{R_{eq}} = \frac{1}{R_1}+\frac{1}{R_2}$
\end{center}
\end{column}
\end{columns}
\end{frame}

\begin{frame}\frametitle{Resistors in Series}
\begin{columns}
\begin{column}{5cm}
\begin{center}
\includegraphics[scale=0.25]{fig/series.png}

\vspace{0.5cm}

Node Law: $I = I_1 = I_2 = I_3$

\vspace{0.2cm}

Loop Law: $V_{cc} - (V_1 + V_2 + V_3) = 0$

\end{center}
\end{column}
\begin{column}{5cm}

Rearranging the Loop Law:
\begin{equation}
V_{cc} = V_1 + V_2 + V_3
\end{equation}

Using Ohm's Law:
\begin{equation}
V_{cc} = I R_1 + I R_2 + I R_3
\end{equation}

Using the Distributive Property:
\begin{equation}
V_{cc} = I (R_1 + R_2 + R_3)
\end{equation}

Gives the Equivalent Resistance:
\begin{equation}
\boxed{R_{eq} = R_1 + R_2 + R_3}
\end{equation}

\end{column}
\end{columns}
\end{frame}

\begin{frame}\frametitle{Resistors in Parallel}
\begin{columns}
\begin{column}{5cm}
\begin{center}
\includegraphics[scale=0.25]{fig/parallel.png}

\vspace{0.5cm}

Node Law: $I = I_1 + I_2$

\vspace{0.2cm}


Loop Law: $V_{cc} = V_1 = V_2$

\end{center}
\end{column}
\begin{column}{5cm}

\begin{equation}
I = I_1 + I_2
\end{equation}

\begin{equation}
I = \frac{V_1}{R_1} + \frac{V_2}{R_2}
\end{equation}

\begin{equation}
I = \frac{V_{cc}}{R_1} + \frac{V_{cc}}{R_2}
\end{equation}

\begin{equation}
\frac{V_{cc}}{R_{eq}} = V_{cc} (\frac{1}{R_1} + \frac{1}{R_2})
\end{equation}

\begin{equation}
\boxed{\frac{1}{R_{eq}} = (\frac{1}{R_1} + \frac{1}{R_2})}
\end{equation}
\end{column}
\end{columns}
\end{frame}

\section{Capacitance and Filters}

\begin{frame}\frametitle{Capacitors}
\begin{columns}
\begin{column}{7cm}
A capacitor is created out of two metal plates and an insulating material called a dielectric. The metal plates are placed very close to each other, in parallel, but the dielectric sits between them to make sure they don't touch.
\begin{itemize}
\item The dielectric can be made out of all sorts of insulating materials; paper, glass, rubber, ceramic, plastic, or anything that will impede the flow of current.
\item The plates are made of a conductive material; aluminum, tantalum, silver, or other metals. 
\end{itemize}
\end{column}
\begin{column}{4cm}
\begin{overprint}
\includegraphics[scale=0.4]{fig/capwich.png}
\end{overprint}
\end{column}
\end{columns}
\end{frame}

\begin{frame}\frametitle{Capacitors}
When current flows into a capacitor, the charges get "stuck" on the plates because they cannot get past the insulating dielectric. Electrons build up on one of the plates, and it becomes overall negatively charged. The large amount of negative charges pushes away like charges on the other plate, making it positively charged.
\begin{figure}
\includegraphics[scale=0.35]{fig/howcap.png} 
\end{figure}
The stationary charges on these plates create an electric field, which influences electric potential energy and voltage. When charges group together on a capacitor like this, the capacitor is storing electric energy just as a battery might store chemical energy.
\end{frame}

\begin{frame}\frametitle{Capacitors in Circuits}
\begin{center}
\includegraphics[scale=0.30]{fig/aliencap1.jpg}
\end{center}
\end{frame}

\begin{frame}\frametitle{Capacitors in Circuits}
\begin{center}
\includegraphics[scale=0.30]{fig/aliencap2.jpg}
\end{center}
\end{frame}

\begin{frame}\frametitle{RC Time Constant}
\begin{columns}
\begin{column}{4cm}
\begin{center}
\includegraphics[scale=0.25]{fig/capacitor.png}

$V_{c}(t) = V_{c}(0) * e^{\frac{t}{-\tau}}$
\end{center}
\end{column}
\begin{column}{7cm}
Capacitance is defined as:
\begin{center}
$C = \frac{Q}{V} (\frac{Coulombs}{Volt})$ 
\end{center}

The current through a capacitor is:
\begin{center}
$I = C \frac{\Delta V}{\Delta t}$
\end{center}

And, therefore, the capacitor charges with a time constant ($\tau$):
\begin{center}
$ \tau = RC $
\includegraphics[scale=0.60]{fig/tau.jpg}
\end{center}
\end{column}
\end{columns}
\end{frame}

\begin{frame}\frametitle{Alternating Current}
\begin{center}
\includegraphics[scale=0.35]{fig/aliencap3.jpg}
\end{center}
\end{frame}

\begin{frame} \frametitle{Low Pass Filter - cutoff frequency $f_{c}$}
\begin{columns}
\begin{column}{7cm}
\begin{itemize}
\item At low frequencies, there is plenty of time for the capacitor to charge up to practically the same voltage as the input voltage.
\item At high frequencies, the capacitor only has time to charge up a small amount before the input switches direction. The output goes up and down only a small fraction of the amount the input goes up and down. At double the frequency, there's only time for it to charge up half the amount.
\end{itemize}
\end{column}
\begin{column}{4cm}
\begin{overprint}
\includegraphics[scale=0.25]{fig/LP_circuit.png}
\end{overprint}
$ f_{c} = \frac{1}{2 \pi \tau} = \frac{1}{2 \pi RC} $
\end{column}
\end{columns}
\end{frame}

\begin{frame}\frametitle{Low Pass Filter Response}
\begin{center}
	\includegraphics[scale=0.35]{fig/LP_bodeplot.png}
$ f_{c} = \frac{1}{2 \pi RC} = \frac{1}{2 \pi (500) (4.7 x 10^{-6})} = 67.5678 Hz $
\end{center}
\end{frame}

\begin{frame}\frametitle{Other types of filters}
\begin{figure}[h]
	\includegraphics[width=9cm]{fig/filters.jpg}
\end{figure}
\end{frame}


\begin{frame} \frametitle{Capacitors - does it matter how they are placed}
\begin{columns}
\begin{column}{6cm}
\begin{itemize}
\item Some types of capacitors (electrolytic and tantalum) are polarized (they have + and - terminals). This is due to how the dielectric film has been deposited. The reverse polarity leads to degradation of the dielectric.
\item Other capacitors (ceramic and film)  do not have a polarity and can be installed in either direction. 
\end{itemize}
\end{column}
\begin{column}{4cm}
\includegraphics[scale=0.5]{fig/cap_pol.png}
\includegraphics[scale=0.5]{fig/cap_gen.png}
\end{column}
\end{columns}
\end{frame}

\section{Switches, Inductance, and Relays}

\begin{frame}\frametitle{Switches}
\begin{figure}[h]
	\includegraphics[scale=0.40]{fig/switchstate2.png}
\end{figure}
\end{frame}

\begin{frame}\frametitle{Poles and Throws}
\begin{figure}[h]
	\includegraphics[scale=0.25]{fig/st2.jpg}
\end{figure}
\begin{itemize}
\item Poles indicates the number of circuits that one switch can control for one operation of the switch. 
\item Throws indicates the number of contact points.
\end{itemize}
\end{frame}


\begin{frame}\frametitle{A Button - SPST}
\begin{figure}[h]
	\includegraphics[scale=0.20]{fig/SPST2.png}
\end{figure}
\end{frame}

\begin{frame}\frametitle{Maxwell's Equations}

\begin{itemize}
\item Gauss's Law
\[ \nabla \cdot E = \frac{\rho}{\epsilon_0}\]
\item Gauss's Law for Magnetism
\[ \nabla \cdot B = 0\]
\item Faraday's Law
\[ \nabla \times E = -\frac{\delta B}{\delta t} \]
\item Ampere-Maxwell Law
\[ \nabla \times B = \mu_0 J + \mu_0 \epsilon_0 \frac{\delta E}{\delta t} \]
\end{itemize}
\end{frame}


\begin{frame}\frametitle{Maxwell's Equations - What They Mean}

\begin{itemize}
\item Gauss's Law: Electric charges are "sources" or "sinks" for electric fields
\item Gauss's Law for Magnetism: There are no magnetic monopoles
\item Faraday's Law: A changing magnetic field induces current
\item Ampere-Maxwell Law: Moving charges and changing electric fields both create circulating magnetic fields. 
\end{itemize}

\vspace{1cm}

Permittivity and Permeability:
\begin{itemize}
\item Permittivity of free space: $\epsilon_0 = 8.85 \times 10^{-12}$ (F/m)
\item Permeability of free space: $\mu_{0} = 4 \pi \times 10^{-7}$ (H/m)
\end{itemize}

\end{frame}


\begin{frame}\frametitle{Inductors}
\begin{columns}
\begin{column}{4cm}
\begin{center}
\includegraphics[scale=0.7]{fig/bsav3.png}

\vspace{0.25cm}

\includegraphics[scale=0.7]{fig/sol3.png}

\vspace{0.25cm}

\includegraphics[scale=1.00]{fig/ind.png}

\end{center}
\end{column}
\begin{column}{6.5cm}
\begin{itemize}
\item Current flowing in a wire produces a magnetic field (B) around the wire (from Ampere's Law).
\item Wire wrapped into a coil produces a magnetic field that resembles a bar magnet through the center of the coil.
\item Also, in a coil, this magnetic field produces an effect known as Inductance (L) that opposes changes in electric current. 
\end{itemize}
\end{column}
\end{columns}
\end{frame}


\begin{frame}\frametitle{Relays}
\begin{columns}
\begin{column}{5.5cm}
\begin{center}
\includegraphics[width=5.5cm]{fig/relayschem.jpg}
\end{center}
\end{column}
\begin{column}{6cm}
\begin{itemize}
\item When the control terminal has current, the coil around the iron core creates a magnetic field
\item The magnetized core attracts a cantilevered arm closing the circuit to the load
\item When current is stopped, the core is no longer magnetized and a spring opens the contact.
\end{itemize}
\end{column}
\end{columns}
\end{frame}

\begin{frame}\frametitle{Optocoupled Relay}
\begin{center}
\includegraphics[scale=0.40]{fig/optorelay.png}
\end{center}
\begin{columns}
\begin{column}{5cm}
\includegraphics[scale=0.15]{fig/relay3va.jpg}
\end{column}
\begin{column}{5cm}
\begin{itemize}
\item Optocoupler isolates the relay load (which could be up to 240V) from the controller electronics.
\end{itemize}
\end{column}
\end{columns}
\end{frame}

\begin{frame}\frametitle{The 35144 Relay in the Vacuum Systems}

Double Pole, Double Throw Relay

\begin{columns}
\begin{column}{5.5cm}
\begin{center}
\includegraphics[width=5.5cm]{fig/relayImage2.jpg}
\end{center}
\end{column}
\begin{column}{5.5cm}
\begin{center}
\includegraphics[width=5.5cm]{fig/relayImage3.jpg}
\end{center}
\end{column}
\end{columns}
\end{frame}

\begin{frame}\frametitle{Contactor}
\begin{columns}
\begin{column}{5.5cm}
\begin{center}
\includegraphics[width=3.5cm]{fig/contactor.jpg}

\vspace{0.25cm}

\includegraphics[width=5.5cm]{fig/contactor2.jpg}
\end{center}
\end{column}
\begin{column}{6cm}
Difference between contactor and other relays
\begin{itemize}
\item Higher current, voltage, and power ratings
\item Often join 2 poles without a common circuit between them
\item Built in arc suppression, allowing them to handle high currents, such as motor starting inrush currents

\begin{itemize}
\item Gas filled Contactors
\item Magnetic Blowout Contactors
\end{itemize}
\end{itemize}


\end{column}
\end{columns}
\end{frame}



\begin{frame}\frametitle{Electrical Safety - Three Wires}
\begin{center}
\includegraphics[scale=0.40]{fig/lng.jpg}
\end{center}
\end{frame}

\begin{frame}\frametitle{Electrical Safety - Electric Shock}
\begin{center}
\includegraphics[scale=0.40]{fig/inductedemf.jpg}
\end{center}
\end{frame}

\begin{frame}\frametitle{Electrical Safety - GFCI}
\begin{center}
\includegraphics[scale=0.80]{fig/gfci.png}
\end{center}
\end{frame}

\begin{frame}\frametitle{Electrical Safety - 240V}
\begin{center}
\includegraphics[scale=1.8]{fig/hse.png}
\end{center}
\end{frame}

\begin{frame}\frametitle{Arc Flash}
\begin{center}
\includegraphics[width=8cm]{fig/arcflash1.jpg}
\end{center}
\end{frame}

\begin{frame}\frametitle{Arc Flash}
\begin{center}
\includegraphics[width=12cm]{fig/arcflash2.jpg}
\end{center}
\end{frame}


\begin{frame}\frametitle{Three Point Check}
\begin{columns}
\begin{column}{4cm}
\begin{center}
\includegraphics[height=6cm]{fig/threePoint.jpg}
\end{center}
\end{column}
\begin{column}{7.5cm}
\begin{enumerate}
\item Verify the test tool works properly when the function switch is placed to "voltage" by testing for voltage on a known energized source, or by using an electronic proving unit, and observing the correct reading on the meter face.
\item Test the circuit to be verified by measuring phase-to-phase and phase-to-ground across all phases. Zero energy must be indicated.
\item Ensure the test tool still indicates voltage properly by placing the test probes, once again, on a known, energized source or the electronic proving unit.
\end{enumerate}

\end{column}
\end{columns}
\end{frame}

\section{Transformers and Rectifiers}

\begin{frame}\frametitle{Magnetic Flux}

\begin{center}	
	\includegraphics[width=4cm]{fig/fluxCap.jpg}
\end{center}

\vspace{.25cm}

Magnetic Flux ($\Phi$) is the amount of magnetic field lines per unit area.

\end{frame}


\begin{frame}\frametitle{Transformer}
\begin{columns}
\begin{column}{5.5cm}
\begin{center}	
	\includegraphics[width=5.2cm]{fig/Bflux2.jpg}
\end{center}
\end{column}
\begin{column}{5.5cm}
From Faraday's Law:

\[ V = -N \frac{\Delta \Phi}{\Delta t} \]

Because the change in flux is the same on both sides of the transformer:

\[\frac{V_s}{V_p} = \frac{N_s}{N_p} \]


The voltage change is a result of the different number of windings

\end{column}
\end{columns}
\end{frame}


\begin{frame}\frametitle{Full Wave Rectifier}

\begin{center}	
	\includegraphics[width=8cm]{fig/rectifier1.jpg}
\end{center}

\end{frame}


\begin{frame}\frametitle{Full Wave Rectifier in Action}
\begin{columns}
\begin{column}{5.5cm}
Positive Half Cycle
\begin{center}	
	\includegraphics[width=5.2cm]{fig/rectifier2.jpg}
\end{center}
\end{column}
\begin{column}{5.5cm}
Negative Half Cycle
\begin{center}	
	\includegraphics[width=5.2cm]{fig/rectifier3.jpg}
\end{center}
\end{column}
\end{columns}
\end{frame}

\begin{frame}\frametitle{Full Wave Rectifier with Smoothing Capacitor}
\begin{center}	
	\includegraphics[width=8cm]{fig/rectifier4.jpg}
\end{center}
\end{frame}

\begin{frame}\frametitle{Putting it All Together}
\begin{center}	
	\includegraphics[width=12cm]{fig/rectifier6.jpg}
\end{center}
\end{frame}

\section{Diodes and Photodetectors}

\begin{frame}\frametitle{Diodes}
The key function of a diode is to control the direction of current-flow. Current passing through a diode can only go in one direction, called the forward direction. Current trying to flow the reverse direction is blocked. \newline 
\begin{columns}
\begin{column}{5cm}
\includegraphics[scale=0.25]{fig/idealdiode.png}
\end{column}
\begin{column}{5cm}
\includegraphics[scale=0.25]{fig/actualdiode.png}
\end{column}
\end{columns}
A non-ideal diode needs a small forward voltage ($V_F$) to turn on and will fail (breakdown) with a large negative voltage ($V_{BR}$).
\end{frame}

\begin{frame}
\frametitle{Light Emitting Diodes}
LEDs (that's "ell-ee-dees") are a particular type of diode that convert electrical energy into light.
\begin{figure}	
	\includegraphics[scale=1.0]{fig/LEDs.png}
\end{figure}
\end{frame}

\begin{frame}\frametitle{Diodes in Reverse and Forward Bias}
\begin{figure}	
	\includegraphics[width=10cm]{fig/diodeFR.jpg}
\end{figure}
\end{frame}

\begin{frame}\frametitle{Photodiode}
\begin{columns}
\begin{column}{5.5cm}
\begin{center}	
	\includegraphics[width=5.2cm]{fig/photodiode3.jpg}
\end{center}
\end{column}
\begin{column}{5.5cm}
\begin{center}	
	\includegraphics[width=5.2cm]{fig/photodiode2.png}
\end{center}
\end{column}
\end{columns}
\end{frame}

\begin{frame}\frametitle{Avalanche Photodiode (APD)}
\begin{columns}
\begin{column}{5cm}
\begin{center}	
	\includegraphics[width=5cm]{fig/apd.jpg}
\end{center}
\end{column}
\begin{column}{6cm}
\begin{itemize}
\item Operates in reverse direction and biased slightly beyond the breakdown threshold voltage $V_{BR}$ (Geiger Mode)
\item An active quenching circuit limits the current through the APD in order to avoid destruction and lowers the bias voltage below $V_{BR}$
\item There is a deadtime after quenching
\item In addition to photo-generated events, Dark Counts can happen spontaneously
\end{itemize}
\end{column}
\end{columns}
\end{frame}

\section{Operational Amplifiers}

\begin{frame}\frametitle{Ideal Op Amp}
\begin{columns}
\begin{column}{4cm}
\begin{center}	
	\includegraphics[width=4cm]{fig/opAmpIdeal.jpg}
	
\vspace{0.25cm}	
	
	\includegraphics[width=4cm]{fig/opAmpIdeal2.png}
\end{center}
\end{column}
\begin{column}{7cm}
\begin{itemize}
\item Op Amp amplifies the difference in voltage between the two inputs
\item Ideal OpAmp Assumptions
\begin{itemize}
\item Amplification is infinite ($> 20k$) 
\[ v_{n} = v_{p} \] 
\item Input impedance is infinite
\[i_{n} = i_{p} = 0\]
\item Output impedance is zero
\end{itemize}
\end{itemize}
\end{column}
\end{columns}
\end{frame}

\begin{frame}\frametitle{OpAmp in saturation}

\begin{center}	
	\includegraphics[width=7cm]{fig/opAmpSat.jpg}
\end{center}

\end{frame}

\begin{frame}\frametitle{Op Amp Circuits}
\begin{columns}
\begin{column}{5.5cm}
\begin{center}	
	\includegraphics[width=5.2cm]{fig/opamp1.jpg}
\end{center}
\end{column}
\begin{column}{5.5cm}
\begin{center}	
	\includegraphics[width=5.2cm]{fig/opamp2.jpg}
\end{center}
\end{column}
\end{columns}
\end{frame}

\begin{frame}\frametitle{Capacitors Revisited}
\begin{columns}
\begin{column}{5.5cm}
\begin{center}	
	\includegraphics[width=3cm]{fig/abp0.png}
\end{center}
\end{column}
\begin{column}{5.5cm}
Capacitor acts like:
\begin{itemize}
\item High Frequency: Short
\item Low Frequency: Open
\end{itemize}
\end{column}
\end{columns}
\end{frame}

\begin{frame}\frametitle{Passive Filters}
\begin{columns}
\begin{column}{5.5cm}
\begin{center}	
	\includegraphics[width=5.5cm]{fig/pbp1.jpg}
\end{center}
\end{column}
\begin{column}{5.5cm}
\begin{center}	
	\includegraphics[width=5.5cm]{fig/pbp2.jpg}
\end{center}
\end{column}
\end{columns}

\vspace{1cm}

\begin{center}
	\includegraphics[width=7cm]{fig/abp3.jpg}
\end{center}

\end{frame}




\begin{frame}\frametitle{Active Bandpass Filter}

\begin{center}	
	\includegraphics[width=8cm]{fig/abp1.jpg}
	
	\vspace{1cm}
	
	\includegraphics[width=8cm]{fig/abp3.jpg}
		
	
\end{center}

\end{frame}


\begin{frame}\frametitle{Active Bandpass Response}
\begin{center}	
	\includegraphics[width=8cm]{fig/abp4.jpg}	
\end{center}
\end{frame}

\begin{frame}\frametitle{Inverting Active Bandpass Filter}
\begin{center}	
	\includegraphics[width=10cm]{fig/abp2.jpg}	
\end{center}

\[f_L = \frac{1}{2 \pi R_1 C_1} \hspace{2cm} f_H = \frac{1}{2 \pi R_2 C_2} \]

\end{frame}

\begin{frame}\frametitle{Non-inverting Bandpass filter}
\begin{columns}
\begin{column}{4.5cm}
\begin{center}	
	\includegraphics[width=4.5cm]{fig/noninvamp_bb.jpg}
\end{center}
\end{column}
\begin{column}{6.5cm}
\begin{center}	
	\includegraphics[width=6.5cm]{fig/noninvamp_schem.jpg}
\end{center}
\end{column}
\end{columns}
\end{frame}


\begin{frame}\frametitle{Higher Order Butterworth (Lowpass) Filters}
\begin{columns}
\begin{column}{4cm}
\begin{center}	
	\includegraphics[width=4cm]{fig/butter2.jpg}
\end{center}
\end{column}
\begin{column}{6.5cm}
\begin{center}	
	\includegraphics[width=6cm]{fig/butter1.jpg}
\end{center}
\end{column}
\end{columns}

\vspace{1cm}

First described in 1930 by the British engineer and physicist Stephen Butterworth in his paper entitled "On the Theory of Filter Amplifiers".

\end{frame}


\end{document}
